\documentclass[11pt]{article}
\usepackage[utf8]{inputenc}
\usepackage[T1]{fontenc}
\usepackage{amsmath, amssymb, amsthm}
\usepackage{geometry}

\geometry{margin=1in}
\newcommand{\N}{\mathbb{N}}
\newcommand{\Nzero}{\mathbb{N} \cup \{0\}}
\newcommand{\Qtwo}{Q_2}
\newcommand{\Z}{\mathbb{Z}}
\newcommand{\F}{\mathbb{F}}
\newcommand{\eps}{\varepsilon}
\DeclareMathOperator{\AR}{AR}

\usepackage[most]{tcolorbox}

% --- Defined Colors ---
\definecolor{boxblue}{RGB}{0, 0, 150}
\definecolor{boxback}{RGB}{245, 245, 255}

% --- The Problem Environment ---
\newtcolorbox{problem}[1]{%
    colback=boxback,
    colframe=boxblue,
    fonttitle=\bfseries\large,
    title={#1},
    sharp corners,
    enhanced,
    attach boxed title to top left={yshift=-2mm, xshift=2mm},
    boxed title style={colframe=boxblue, colback=boxblue},
    before skip=15pt plus 2pt,
    after skip=15pt plus 2pt,
    top=10pt, bottom=10pt, left=10pt, right=10pt
}

\newtcolorbox{solution}[1]{%
    colback=white,
    colframe=boxblue,
    fonttitle=\bfseries\large,
    title={#1},
    sharp corners,
    enhanced jigsaw, % Better frame handling for page breaks than just 'enhanced'
    breakable,       % <--- Allows the box to split across pages
    attach boxed title to top left={yshift=-2mm, xshift=2mm},
    boxed title style={colframe=boxblue, colback=boxblue},
    before skip=15pt plus 2pt,
    after skip=15pt plus 2pt,
    top=10pt, bottom=10pt, left=10pt, right=10pt
}



\newtheorem{theorem}{Theorem}
\newtheorem*{theorem*}{Theorem}
\newtheorem{lemma}{Lemma}
\newtheorem{definition}{Definition}
\newtheorem{proposition}[theorem]{Proposition}
\newtheorem*{remark}{Contextual Remarks}

\begin{document}


\emph{The prompts and responses below have been edited only for
formatting compatibility. NOTE THAT MANY CONTAIN MINOR INACCURACIES.}

\begin{problem}{Erd\H{o}s-75}
Is there a graph of chromatic number $\aleph_1$ such that for all $\epsilon>0$ if $n$ is sufficiently large and $H$ is a subgraph on $n$ vertices then $H$ contains an independent set of size $>n^{1-\epsilon}$?
\end{problem}

\begin{solution}{Model response (\emph{Aletheia})}
\begin{theorem}
There exists a graph $G$ with chromatic number $\aleph_1$ such that for all $\epsilon > 0$, if $n$ is sufficiently large and $H$ is a subgraph of $G$ on $n$ vertices, then $H$ contains an independent set of size greater than $n^{1-\epsilon}$.
\end{theorem}

\begin{proof}
The proof relies on a result by Chris Lambie-Hanson regarding the existence of uncountably chromatic graphs whose finite subgraphs have arbitrarily slowly growing chromatic numbers relative to their size.

\paragraph{1. Lambie-Hanson's Theorem}
The following theorem was proven in ZFC (Zermelo–Fraenkel set theory with Choice):

\begin{quote}
\textbf{Theorem [Lambie-Hanson, 2020]:} For every function $f: \mathbb{N} \to \mathbb{N}$, there exists a graph $G$ with chromatic number $\chi(G) = \aleph_1$ such that for every integer $k \ge 3$, every subgraph of $G$ with chromatic number at least $k$ has at least $f(k-3)$ vertices.
\end{quote}

\textit{Reference:} C. Lambie-Hanson, ``On the growth rate of chromatic numbers of finite subgraphs,'' \textit{Advances in Mathematics}, vol. 369, Art. 107176, 2020.

\paragraph{2. Construction of the Graph}
Let $f: \mathbb{N} \to \mathbb{N}$ be the double exponential function defined by:
\[ f(x) = 2^{2^x}. \]
Applying Lambie-Hanson's Theorem with this specific function $f$, there exists a graph $G$ such that $\chi(G) = \aleph_1$, and for any subgraph $H \subseteq G$, if $\chi(H) = k \ge 3$, then $|V(H)| \ge 2^{2^{k-3}}$.

\paragraph{3. Analysis of Finite Subgraphs}
Let $\epsilon > 0$ be an arbitrary positive constant. Consider any subgraph $H$ of $G$ with $n$ vertices, where $n$ is sufficiently large (specifically, large enough to satisfy the inequality derived below).

Let $k = \chi(H)$ be the chromatic number of $H$. If $k < 3$, then $\chi(H) \le 2$ and $\alpha(H) \ge n/2$, which is greater than $n^{1-\epsilon}$ for large $n$. Thus, we assume $k \ge 3$.
According to the construction of $G$, the number of vertices $n$ must satisfy:
\[ n \ge 2^{2^{k-3}}. \]
Taking logarithms base 2 twice yields:
\[ \log_2(\log_2 n) \ge k - 3, \]
which provides an upper bound for the chromatic number of $H$:
\[ \chi(H) \le \log_2(\log_2 n) + 3. \]

We recall the elementary inequality relating the independence number $\alpha(H)$ and chromatic number $\chi(H)$ for a graph with $n$ vertices:
\[ \alpha(H) \ge \frac{n}{\chi(H)}. \]
Substituting our upper bound for $\chi(H)$:
\[ \alpha(H) \ge \frac{n}{\log_2(\log_2 n) + 3}. \]

\paragraph{4. Verification of the Independent Set Condition}
We require that $\alpha(H) > n^{1-\epsilon}$ for all sufficiently large $n$. Comparing our lower bound for $\alpha(H)$ to the desired value:
\[ \frac{\alpha(H)}{n^{1-\epsilon}} \ge \frac{n / (\log_2(\log_2 n) + 3)}{n^{1-\epsilon}} = \frac{n^\epsilon}{\log_2(\log_2 n) + 3}. \]
Because $\epsilon$ is a fixed positive constant, the polynomial term $n^\epsilon$ grows strictly faster than the iterated logarithmic term as $n \to \infty$. Formally:
\[ \lim_{n \to \infty} \frac{n^\epsilon}{\log_2(\log_2 n) + 3} = \infty. \]
Thus, there exists an integer $N_\epsilon$ such that for all $n > N_\epsilon$,
\[ \frac{n^\epsilon}{\log_2(\log_2 n) + 3} > 1 \implies \alpha(H) > n^{1-\epsilon}. \]

This establishes that for the chosen graph $G$, for \textit{any} $\epsilon > 0$, every sufficiently large finite subgraph $H$ contains an independent set of the required size.
\end{proof}

\subsubsection*{Conclusion}
Yes, such a graph exists. By utilizing Lambie-Hanson's construction in ZFC, we obtain a graph of chromatic number $\aleph_1$ where the chromatic number of finite subgraphs grows so slowly (at most double-logarithmically in vertex count) that the independence number $\alpha(H)$ asymptotically exceeds $n^{1-\epsilon}$ for every $\epsilon > 0$.
\end{solution}

\begin{solution}{Model response (internal proprietary agent)}
Yes, such a graph exists. A standard example is the \textbf{shift graph} defined on the cardinal $\kappa = (2^{\aleph_0})^+$.

This graph, denoted here as $G$, has chromatic number exactly $\aleph_1$. Furthermore, every subgraph $H$ of $G$ with $n$ vertices satisfies $\alpha(H) \ge n/4$. Consequently, for any $\epsilon > 0$ and sufficiently large $n$, the independence number exceeds $n^{1-\epsilon}$.

\section{Definition of the Graph}

Let $\kappa = (2^{\aleph_0})^+$. We define the graph $G = (V, E)$ as follows:

\begin{itemize}
    \item \textbf{Vertices:} The set of all ordered pairs of distinct ordinals in $\kappa$.
    \[ V = [\kappa]^2 = \bigl\{ \{\alpha, \beta\} : \alpha, \beta \in \kappa \text{ and } \alpha < \beta \bigr\} \]
    \item \textbf{Edges:} Two vertices are adjacent if the larger element of one pair coincides with the smaller element of the other. That is, for any $\alpha < \beta < \gamma$:
    \[ \{\alpha, \beta\} \sim \{\beta, \gamma\} \]
\end{itemize}

\textit{Note: While the vertex set of this specific graph has cardinality $(2^{\aleph_0})^+$, the Erdős–Hajnal theorem implies that $G$ contains a subgraph with exactly $\aleph_1$ vertices that retains the chromatic number $\aleph_1$. The properties proved below hold for $G$ and any of its subgraphs.}

\section{Proof that $\chi(G) = \aleph_1$}

We determine the chromatic number by proving both lower and upper bounds.

\subsection*{Lower Bound: $\chi(G) > \aleph_0$}

Assume for the sake of contradiction that there exists a proper coloring $c: V \to \omega$ utilizing countably many colors. This is equivalent to a coloring of the pairs of $\kappa$ with $\aleph_0$ colors. We invoke the \textbf{Erdős–Rado Partition Relation}:
\[ (2^{\aleph_0})^+ \to (\aleph_1)^2_{\aleph_0} \]
This partition relation asserts that for any coloring $c: [\kappa]^2 \to \omega$, there exists a subset $Y \subseteq \kappa$ of cardinality $\aleph_1$ that is \textit{homogeneous}, meaning all pairs formed from elements of $Y$ are assigned the same color, say $k$.

Let $\alpha < \beta < \gamma$ be three distinct ordinals from $Y$. Consider the graph vertices $u = \{\alpha, \beta\}$ and $v = \{\beta, \gamma\}$.
\begin{enumerate}
    \item Since $\alpha, \beta, \gamma \in Y$, both pairs $u$ and $v$ belong to $[Y]^2$. Thus, $c(u) = k$ and $c(v) = k$.
    \item By the definition of edges in $G$, $u$ and $v$ are adjacent because $\max(u) = \beta = \min(v)$.
\end{enumerate}

We have found two adjacent vertices with the same color. This contradicts the assumption that $c$ is a proper coloring. Therefore, $G$ cannot be colored with countably many colors: $\chi(G) \ge \aleph_1$.

\subsection*{Upper Bound: $\chi(G) \le \aleph_1$}

Since $|\kappa| = (2^{\aleph_0})^+ \le 2^{\aleph_1}$ (by Cantor's theorem $2^{\aleph_0} < 2^{\aleph_1}$), there exists an injective mapping $\phi: \kappa \to \{0, 1\}^{\aleph_1}$. We define a coloring $C$ using the set of colors $\aleph_1 \times \{0, 1\}$, which has cardinality $\aleph_1$.

For any vertex $u = \{\alpha, \beta\}$ with $\alpha < \beta$, let $\delta$ be the smallest ordinal index where the sequences $\phi(\alpha)$ and $\phi(\beta)$ differ. Define:
\[ C(u) = (\delta, \phi(\alpha)_\delta) \]

To verify that $C$ is proper, suppose two adjacent vertices $u = \{\alpha, \beta\}$ and $v = \{\beta, \gamma\}$ have the same color $(\delta, i)$.
\begin{enumerate}
    \item For $u$: Since $C(u) = (\delta, i)$, we have $\phi(\alpha)_\delta = i$. Because $\delta$ is the index of the first difference between $\phi(\alpha)$ and $\phi(\beta)$, it follows that $\phi(\beta)_\delta \neq i$. Thus, $\phi(\beta)_\delta = 1 - i$.
    \item For $v$: Since $C(v) = (\delta, i)$, the value of the sequence of its \textit{first} component (which is $\beta$) at index $\delta$ must be $i$. Thus, $\phi(\beta)_\delta = i$.
\end{enumerate}

These two conditions imply $1 - i = i$, which is impossible for $i \in \{0, 1\}$. Thus, $\chi(G) \le \aleph_1$. Combining the inequalities yields $\chi(G) = \aleph_1$.

\section{Independence Number of Finite Subgraphs}

Let $H$ be any subgraph of $G$ with $n$ vertices. We show that the independence number $\alpha(H)$ is at least $n/4$. Let $V(H)$ be the set of vertices in the subgraph and $S$ be the set of all ordinals appearing in the pairs constituting $V(H)$.

Using the probabilistic method, partition $S$ into two sets $A$ and $B$ by assigning each ordinal in $S$ to $A$ or $B$ uniformly and independently with probability $1/2$. Define a subset $I \subseteq V(H)$ as:
\[ I = \bigl\{ \{\alpha, \beta\} \in V(H) : \alpha \in A \text{ and } \beta \in B \bigr\} \]

\textbf{Proof that $I$ is an independent set:}
Let $u, v \in I$ be two vertices. If they were adjacent, their union would be of the form $\{\alpha, \beta, \gamma\}$ with $\alpha < \beta < \gamma$.
\begin{itemize}
    \item Since $\{\alpha, \beta\} \in I$, its second element $\beta$ must be in $B$.
    \item Since $\{\beta, \gamma\} \in I$, its first element $\beta$ must be in $A$.
\end{itemize}
This requires $\beta \in A \cap B$, which is impossible as $A$ and $B$ are disjoint.

\textbf{Expected Size of $I$:}
For any vertex $w = \{\alpha, \beta\} \in V(H)$, $P(w \in I) = P(\alpha \in A)P(\beta \in B) = 1/4$. By linearity of expectation, $\mathbb{E}[|I|] = n/4$. Thus, there exists a partition such that $|I| \ge n/4$, implying $\alpha(H) \ge n/4$.

\section{Verification of the Asymptotic Condition}

Given $\epsilon > 0$, we check if $\alpha(H) > n^{1-\epsilon}$ holds for sufficiently large $n$. Since $\alpha(H) \ge n/4$, we solve:
\[ \frac{n}{4} > n^{1-\epsilon} \implies n^\epsilon > 4 \implies n > 4^{1/\epsilon} \]
Let $N = \lfloor 4^{1/\epsilon} \rfloor$. For all $n > N$, $\alpha(H) > n^{1-\epsilon}$.

\section*{Conclusion}

The shift graph $G$ on $(2^{\aleph_0})^+$ satisfies $\chi(G) = \aleph_1$ and $\alpha(H) \ge n/4$, fulfilling all stated requirements.
\end{solution}

\begin{problem}{User prompt (Erd\H{o}s-333)}
Let $A\subseteq \mathbb{N}$ be a set of density zero. Does there exist a basis $B$ such that $A\subseteq B+B$ and\[\lvert B\cap \{1,\ldots,N\}\rvert =o(N^{1/2})\]for all large $N$?
\end{problem}

\begin{solution}{Model response (\textit{Aletheia})}
Let $A$ be a subset of the natural numbers $\N$. A set $B \subseteq \Nzero$ is called a \textit{basis} for $A$ if every element of $A$ can be expressed as the sum of two elements of $B$. That is,
\[
A \subseteq B+B = \{ b_1 + b_2 : b_1, b_2 \in B \}.
\]
A classic result in additive combinatorics states that if $A = \N$, the counting function $B(N) = |B \cap \{1, \ldots, N\}|$ must satisfy $B(N) \ge \sqrt{N}$. A natural question, appearing as Problem 333 in Erdős and Graham \cite{ErdosGraham}, is whether "thin" sets (those with asymptotic density zero) always possess sparse bases.

\begin{theorem} \label{thm:main}
There exists a set $A \subseteq \N$ of asymptotic density zero such that for every basis $B \subseteq \Nzero$ of $A$, the counting function $B(N)$ satisfies
\[
\limsup_{N \to \infty} \frac{B(N)}{\sqrt{N}} > 0.
\]
In particular, no basis for $A$ satisfies $B(N) = o(N^{1/2})$.
\end{theorem}

\section{Construction of the Set $A$}

Define a sequence of integers $N_k = 4^k$ for $k \ge 1$. Let $\{I_k\}_{k=1}^\infty$ be a sequence of disjoint intervals:
\[
I_k = \{ n \in \N : \frac{N_k}{2} < n \le N_k \}.
\]
Fix $\lambda = 0.1$. Let $\mathcal{F}_k$ be the collection of all subsets $S \subseteq \{0, 1, \ldots, N_k\}$ with $|S| \le \lambda \sqrt{N_k}$. Let $L_k = \lfloor \sqrt{N_k} \ln N_k \rfloor$.

\begin{lemma} \label{lemma:prob_existence}
There exists an integer $K$ such that for all $k \ge K$, there exists a set $A_k \subseteq I_k$ satisfying $|A_k| \le L_k$ and $A_k \not\subseteq S+S$ for all $S \in \mathcal{F}_k$.
\end{lemma}

\begin{proof}
Let $k$ be fixed and $m = \lfloor \lambda \sqrt{N_k} \rfloor$. The size of the family is bounded by:
\[
|\mathcal{F}_k| = \sum_{j=0}^m \binom{N_k+1}{j} \le (m+1) \left( \frac{e(N_k+1)}{m} \right)^m.
\]
Taking logarithms, we find $\ln |\mathcal{F}_k| \le \frac{\lambda}{2} \sqrt{N_k} \ln N_k + O(\sqrt{N_k})$.

For any $S \in \mathcal{F}_k$, the sumset $S+S$ has size at most $\binom{m+1}{2} \approx \frac{\lambda^2 N_k}{2}$. The probability $p$ that a random element $x \in I_k$ falls in $S+S$ is $p = \frac{|(S+S) \cap I_k|}{|I_k|} \le \lambda^2 (1 + o(1))$.

The probability that a random multiset $X$ of size $L_k$ is covered by some $S+S$ is bounded by the union bound:
\[
P_{\text{fail}} \le |\mathcal{F}_k| p^{L_k} \implies \ln P_{\text{fail}} \le \sqrt{N_k} \ln N_k \left( \frac{\lambda}{2} + 2 \ln \lambda \right) + o(\sqrt{N_k} \ln N_k).
\]
With $\lambda = 0.1$, the term in parentheses is approximately $-4.55$, so $P_{\text{fail}} \to 0$. Thus, a suitable $A_k$ exists for large $k$.
\end{proof}

We define our counterexample as $A = \bigcup_{k=K}^{\infty} A_k$.

\section{Density and Basis Growth}

\subsection{Asymptotic Density}
Since $|A_k| \le \sqrt{N_k} \ln N_k$, the cumulative size $|A \cap [1, x]|$ is dominated by the largest $N_k \le x$.
\[
\frac{|A \cap [1, x]|}{x} \le \frac{\sum_{j=K}^k (2 \ln 2) j 2^j}{2^{2k}} \approx \frac{k 2^k}{2^{2k}} = \frac{k}{2^k} \to 0.
\]
Thus, $d(A) = 0$.

\subsection{Proof of the Main Result}
Assume $B$ is a basis for $A$ with $B(N) = o(\sqrt{N})$. For large $k$, let $S^* = B \cap [0, N_k]$. Then $|S^*| \le B(N_k) + 1$. Since $B(N_k) = o(\sqrt{N_k})$, for sufficiently large $k$ we must have $|S^*| < \lambda \sqrt{N_k}$, meaning $S^* \in \mathcal{F}_k$.

Since $A_k \subseteq A$ and $A \subseteq B+B$, any $a \in A_k$ must satisfy $a = b_1 + b_2$ for $b_1, b_2 \in B$. Since $a \le N_k$ and $b_i \ge 0$, it follows $b_i \in [0, N_k]$, so $b_1, b_2 \in S^*$. This implies $A_k \subseteq S^* + S^*$, which contradicts the construction of $A_k$ in Lemma \ref{lemma:prob_existence}.

Therefore, $B(N) \neq o(\sqrt{N})$.

\begin{thebibliography}{9}
\bibitem{ErdosGraham}
P. Erdős and R. L. Graham, \textit{Old and New Problems and Results in Combinatorial Number Theory}, Monographies de L'Enseignement Mathématique, Vol. 28, Geneva, 1980.
\bibitem{ErdosNewman}
P. Erdős and D. J. Newman, \textit{Bases for sets of integers}, Journal of Number Theory \textbf{9} (1977), 420--425.
\end{thebibliography}
\end{solution}


\begin{problem}{User prompt (Erd\H{o}s-397)}
Are there only finitely many solutions to\[\prod_i \binom{2m_i}{m_i}=\prod_j \binom{2n_j}{n_j}\]with the $m_i,n_j$ distinct?
\end{problem}

\begin{solution}{Model response (\text{Aletheia}s)}
The answer to the question of whether such solutions are finite is \textbf{No}; there are infinitely many solutions.

\begin{theorem}
There exist infinitely many distinct pairs of disjoint finite sets of positive integers $(A, B)$ such that
\[ \prod_{m \in A} \binom{2m}{m} = \prod_{n \in B} \binom{2n}{n}. \]
\end{theorem}

\begin{proof}
Let $C_k = \binom{2k}{k}$ denote the $k$-th central binomial coefficient. To prove the existence of infinitely many solutions, we will explicitly construct an infinite family of disjoint sets satisfying the required product identity.

For every integer $k \ge 3$, define the sets $A_k$ and $B_k$ as follows:
\begin{align*}
A_k &= \{k, \, 2k-2, \, 8k^2-8k+2\} \\
B_k &= \{k-1, \, 2k, \, 8k^2-8k+1\}
\end{align*}

We assert two properties for all $k \ge 3$:
\begin{enumerate}
    \item \textbf{Disjointness and Distinctness:} The sets $A_k$ and $B_k$ consist of strictly positive distinct integers and are disjoint.
    \item \textbf{Product Equality:} The product of central binomial coefficients indexed by $A_k$ equals the product of those indexed by $B_k$.
\end{enumerate}

Since the largest element of these sets, $8k^2-8k+2$, is strictly increasing for $k \ge 3$, each $k$ generates a unique solution pair. Thus, establishing these properties proves the theorem.

\subsection*{1. Proof of Distinctness and Disjointness}

We verify that all six indices appearing in $A_k \cup B_k$ are distinct integers. Let us order the elements by magnitude. We claim:
\[ k-1 < k < 2k-2 < 2k < 8k^2-8k+1 < 8k^2-8k+2. \]
Proof of the inequalities for $k \ge 3$:
\begin{enumerate}
    \item $k-1 < k$: Trivial.
    \item $k < 2k-2$: Rearranging gives $2 < k$, which is true by assumption.
    \item $2k-2 < 2k$: Trivial.
    \item $2k < 8k^2-8k+1$: This is equivalent to $8k^2-10k+1 > 0$. The roots of $8x^2-10x+1=0$ are $x = \frac{5 \pm \sqrt{17}}{8}$. Since $\frac{5+\sqrt{17}}{8} \approx 1.14$, the quadratic is positive for all $k \ge 2$.
    \item $8k^2-8k+1 < 8k^2-8k+2$: Trivial.
\end{enumerate}

Also, since $k \ge 3$, the smallest element $k-1 \ge 2$, ensuring all elements are positive integers. Because all elements in the union $A_k \cup B_k$ are distinct, the sets must be disjoint ($A_k \cap B_k = \emptyset$).

\subsection*{2. Proof of the Identity}

The condition $\prod_{m \in A_k} C_m = \prod_{n \in B_k} C_n$ holds if and only if the product of the ratios is $1$. Rearranging terms, we verify:
\[ \left(\frac{C_k}{C_{k-1}}\right) \cdot \left(\frac{C_{2k-2}}{C_{2k}}\right) \cdot \left(\frac{C_{8k^2-8k+2}}{C_{8k^2-8k+1}}\right) = 1. \]
Recall the ratio formula for consecutive central binomial coefficients:
\[ \frac{C_x}{C_{x-1}} = \frac{(2x)!}{(x!)^2} \cdot \frac{((x-1)!)^2}{(2x-2)!} = \frac{2x(2x-1)}{x^2} = \frac{2(2x-1)}{x}. \]

\textbf{Factor 1:} Let $x=k$.
\[ F_1 = \frac{C_k}{C_{k-1}} = \frac{2(2k-1)}{k}. \]

\textbf{Factor 2:} We compute $\frac{C_{2k-2}}{C_{2k}}$ as the reciprocal of the product of two consecutive steps.
\[ \frac{C_{2k}}{C_{2k-2}} = \frac{C_{2k}}{C_{2k-1}} \cdot \frac{C_{2k-1}}{C_{2k-2}} = \left(\frac{2(4k-1)}{2k}\right) \left(\frac{2(4k-3)}{2k-1}\right) = \frac{2(4k-1)(4k-3)}{k(2k-1)}. \]
Inverting this gives:
\[ F_2 = \frac{k(2k-1)}{2(4k-1)(4k-3)}. \]

\textbf{Factor 3:} Let $N = 8k^2-8k+2 = 2(2k-1)^2$. The ratio is:
\[ F_3 = \frac{2(2N-1)}{N} = \frac{2(2(2(2k-1)^2)-1)}{2(2k-1)^2} = \frac{4(2k-1)^2 - 1}{(2k-1)^2}. \]
Using the difference of squares $a^2-b^2 = (a-b)(a+b)$ where $a=2(2k-1)$:
\[ F_3 = \frac{(2(2k-1)-1)(2(2k-1)+1)}{(2k-1)^2} = \frac{(4k-3)(4k-1)}{(2k-1)^2}. \]

\textbf{Verification of Product:}
\[ \text{Product} = \left( \frac{2(2k-1)}{k} \right) \cdot \left( \frac{k(2k-1)}{2(4k-1)(4k-3)} \right) \cdot \left( \frac{(4k-1)(4k-3)}{(2k-1)^2} \right) = \frac{2k(2k-1)^2(4k-1)(4k-3)}{2k(2k-1)^2(4k-1)(4k-3)} = 1. \]
\end{proof}

\subsection*{Example ($k=3$)}
For $k=3$: $A_3 = \{3, 4, 50\}$ and $B_3 = \{2, 6, 49\}$.
\begin{align*}
\prod_{m \in A_3} \binom{2m}{m} &= \binom{6}{3} \binom{8}{4} \binom{100}{50} = 20 \cdot 70 \cdot \binom{100}{50} = 1400 \binom{100}{50}. \\
\prod_{n \in B_3} \binom{2n}{n} &= \binom{4}{2} \binom{12}{6} \binom{98}{49} = 6 \cdot 924 \cdot \binom{98}{49} = 5544 \binom{98}{49}.
\end{align*}
The ratio $\frac{1400 \binom{100}{50}}{5544 \binom{98}{49}} = \frac{1400}{5544} \cdot \frac{2(99)}{50} = \frac{28 \cdot 198}{5544} = \frac{5544}{5544} = 1$.
\end{solution}

\begin{problem}{Erd\H{o}s-591}
    Let $\alpha$ be the infinite ordinal $\omega^{\omega^2}$. Is it true that in any red/blue colouring of the edges of $K_\alpha$ there is either a red $K_\alpha$ or a blue $K_3$?
\end{problem}

\begin{solution}{Model response (\emph{Aletheia})}
    \textbf{Theorem.} The partition relation $\omega^{\omega^2} \to (\omega^{\omega^2}, 3)^2$ holds. That is, for any coloring of the edges of the complete graph on $\omega^{\omega^2}$ with two colors (red and blue), there exists either a red complete subgraph of order type $\omega^{\omega^2}$ or a blue complete subgraph of size 3.

\begin{proof}
We establish the validity of the relation by invoking the characterization of countable partition ordinals derived by Rene Schipperus and Carl Darby.

\subsection*{1. Representation of the Ordinal}
Let $\alpha = \omega^{\omega^2}$. We express $\alpha$ in the standard form used for partition ordinals:
\[ \alpha = \omega^{\omega^\beta} \]
Comparing exponents, we identify $\beta = 2$.

\subsection*{2. Characterization of Countable Partition Ordinals}
The set of countable ordinals $\alpha$ that satisfy the partition relation $\alpha \to (\alpha, 3)^2$ was characterized by Schipperus (2010) \cite{schipperus2010}.

\textit{Theorem (Schipperus):}
For a countable ordinal $\beta$, the relation $\omega^{\omega^\beta} \to (\omega^{\omega^\beta}, 3)^2$ holds if and only if $\beta$ satisfies one of the following conditions:
\begin{enumerate}
    \item $\beta$ is additively indecomposable.
    \item $\beta$ is the sum of exactly two additively indecomposable ordinals, i.e., $\beta = \gamma + \delta$ where $\gamma, \delta$ are indecomposable and $\gamma \ge \delta \ge 1$.
\end{enumerate}

\subsection*{3. Verification for $\beta = 2$}
We test the ordinal $\beta = 2$ against the stated conditions. Recall that an ordinal is additively indecomposable if it is a power of $\omega$ (including $\omega^0 = 1$). We decompose $\beta$:
\[ 2 = 1 + 1 \]
Here, $\gamma = 1$ and $\delta = 1$. Since $1 = \omega^0$, both $\gamma$ and $\delta$ are additively indecomposable. The inequality condition requires $\gamma \ge \delta \ge 1$. Substituting the values, we have $1 \ge 1 \ge 1$, which is clearly true.

Thus, $\beta = 2$ satisfies Condition 2 of Schipperus's Theorem.

\subsection*{4. Historical Confirmation}
Before Schipperus's general characterization, this specific case was established by Carl Darby. In his work on ordinal partition relations, Darby proved that $\omega^{\omega^2}$ satisfies the partition property for pairs into 3 colors. As stated in the \textit{Handbook of Set Theory} \cite{hajnal2010}:
\begin{quote}
    ``Darby (unpublished) proved $\omega^{\omega^2} \to (\omega^{\omega^2}, 3)^2$ about the same time as Schipperus proved his theorem.''
\end{quote}
Additionally, Darby and Larson extended this result to show that the relation actually holds for target size 4:
\[ \omega^{\omega^2} \to (\omega^{\omega^2}, 4)^2 \]
Since the existence of a blue $K_4$ implies the existence of a blue $K_3$, the positive result for 3 is a direct consequence of this stronger theorem.

\subsection*{5. Negative Results for Higher Values}
To contextualize the result, it is important to note that the relation does not hold for all finite integers on the right side. Specifically, for $m=5$:
\[ \omega^{\omega^2} \not\to (\omega^{\omega^2}, 5)^2 \]
This failure occurs because the bound for $\omega^{\omega^\beta}$ where $\beta$ is a sum of $n$ indecomposable ordinals is $2^n$. Since $2 = 1+1$ ($n=2$), the relation holds for $k \le 2^2 = 4$, but fails for $k > 4$. Thus, the question regarding the blue $K_3$ falls strictly within the range of validity.

\subsection*{Conclusion}
Since $\beta=2$ satisfies the necessary and sufficient conditions for the partition relation, the statement is true.
\end{proof}

\begin{thebibliography}{9}
\bibitem{schipperus2010} Schipperus, R. (2010). Countable partition ordinals. \textit{Annals of Pure and Applied Logic}, 161(10), 1195--1215.
\bibitem{hajnal2010} Hajnal, A., \& Larson, J. A. (2010). Partition Relations. In M. Foreman \& A. Kanamori (Eds.), \textit{Handbook of Set Theory} (Vol. 2, pp. 1229--1294). Springer.
\bibitem{darby1999} Darby, C. (1999). \textit{Negative Partition Relations for Ordinals}. Journal of Combinatorial Theory, Series B, 76, 205--222.
\end{thebibliography}
\end{solution}

\begin{problem}{User prompt (Erd\H{o}s-652)}
Let $x_1,\ldots,x_n\in \mathbb{R}^2$ and let $R(x_i)=\#\{ \lvert x_j-x_i\rvert : j\neq i\}$, where the points are ordered such that\[R(x_1)\leq \cdots \leq R(x_n).\]Let $\alpha_k$ be minimal such that, for all large enough $n$, there exists a set of $n$ points with $R(x_k)<\alpha_kn^{1/2}$. Is it true that $\alpha_k\to \infty$ as $k\to \infty$?
\end{problem}


\begin{solution}{Model response (\emph{Aletheia})}
The problem asks whether $\alpha_k \to \infty$ as $k \to \infty$, where $\alpha_k$ is the minimal value such that for sufficiently large $n$, there exists a set of $n$ points in $\mathbb{R}^2$ for which the $k$-th point (ordered by the number of distinct distances) determines fewer than $\alpha_k n^{1/2}$ distinct distances.

\textbf{Theorem.} \textit{Let $x_1, \ldots, x_n \in \mathbb{R}^2$ be a set of $n$ distinct points, ordered such that $R(x_1) \leq \cdots \leq R(x_n)$, where $R(x_i) = \#\{ |x_j - x_i| : j \neq i \}$. Let $\alpha_k$ be the minimal value such that for all sufficiently large $n$, there exists a configuration of $n$ points satisfying $R(x_k) < \alpha_k n^{1/2}$. Then $\lim_{k \to \infty} \alpha_k = \infty$. Specifically, $\alpha_k = \Omega(\sqrt{k})$.}

\begin{proof}
We establish a lower bound on $\alpha_k$ by utilizing incidence geometry, specifically the upper bound on incidences between points and circles in the plane.

Let $k$ be a fixed positive integer. By the definition of $\alpha_k$, for any $\epsilon > 0$, there exists an integer $N_0$ such that for all $n \geq N_0$, there exists a set $P_n = \{x_1, \ldots, x_n\} \subset \mathbb{R}^2$ satisfying:
\[ R(x_k) < (\alpha_k + \epsilon) n^{1/2}. \]
Since the points are ordered by non-decreasing distinct distance counts $R(x_i)$, it follows that for all $i \in \{1, \ldots, k\}$:
\[ R(x_i) \leq R(x_k) < (\alpha_k + \epsilon) n^{1/2}. \]
Let $C = \alpha_k + \epsilon$. Let $S = \{x_1, \ldots, x_k\}$ be the subset of the first $k$ points of $P_n$. For each $x_i \in S$, let $D_i$ denote the set of distinct distances from $x_i$ to the other points in $P_n$. That is, $D_i = \{ |p - x_i| : p \in P_n \setminus \{x_i\} \}$. We have $|D_i| = R(x_i) < C n^{1/2}$.

We construct a family of circles $\mathcal{C}$ defined by these distances:
\[ \mathcal{C} = \bigcup_{i=1}^k \{ \Gamma(x_i, r) \mid r \in D_i \}, \]
where $\Gamma(x_i, r)$ is the circle centered at $x_i$ with radius $r$.
All circles in $\mathcal{C}$ are distinct because:
\begin{enumerate}
    \item Circles with the same center $x_i$ have distinct radii $r \in D_i$.
    \item Circles with different centers $x_i, x_j$ are distinct regardless of radii.
\end{enumerate}
Therefore, the number of distinct circles in $\mathcal{C}$ is:
\[ |\mathcal{C}| = \sum_{i=1}^k |D_i| < k C n^{1/2}. \]

Next, we estimate the number of incidences $I(P_n, \mathcal{C})$. Consider any point $p \in P_n \setminus S$. For every center $x_i \in S$, the distance $r = |p - x_i|$ is in $D_i$. Thus, $p$ lies on the circle $\Gamma(x_i, r) \in \mathcal{C}$. Every point $p \in P_n \setminus S$ is incident to at least $k$ circles in $\mathcal{C}$. Consequently:
\[ I(P_n, \mathcal{C}) \geq (n - k) k = nk - k^2. \]

We now apply the Pach-Sharir incidence bound \cite{pach1992}.
\textbf{Theorem (Pach \& Sharir, 1992).} \textit{Let $P$ be a set of $m$ points and $\mathcal{L}$ be a set of $L$ distinct simple curves in the plane such that any two curves intersect in at most $s$ points. Then there exists a constant $K_s$ such that:}
\[ I(P, \mathcal{L}) \leq K_s \left( m^{2/3} L^{2/3} + m + L \right). \]
Applying this with $s=2$, $m=n$, and $L = |\mathcal{C}| < k C n^{1/2}$:
\[ I(P_n, \mathcal{C}) \leq K_2 \left( n^{2/3} (k C n^{1/2})^{2/3} + n + k C n^{1/2} \right). \]
Simplifying the first term gives $n k^{2/3} C^{2/3}$. Combining the bounds:
\[ nk - k^2 \leq K_2 \left( n k^{2/3} C^{2/3} + n + k C n^{1/2} \right). \]
Dividing by $n$ and taking the limit as $n \to \infty$:
\[ k \leq K_2 \left( k^{2/3} \alpha_k^{2/3} + 1 \right). \]
Rearranging for $\alpha_k$ (assuming $k$ is large enough such that $k - K_2 \geq k/2$):
\[ \frac{k}{2 K_2} \leq k^{2/3} \alpha_k^{2/3} \implies \frac{k^{1/3}}{2 K_2} \leq \alpha_k^{2/3}. \]
Raising to the power $3/2$:
\[ \alpha_k \geq \left(\frac{1}{2 K_2}\right)^{3/2} \sqrt{k}. \]
Thus, $\alpha_k = \Omega(\sqrt{k})$, which implies $\lim_{k \to \infty} \alpha_k = \infty$.
\end{proof}

\begin{thebibliography}{9}
\bibitem{pach1992}
Pach, J., \& Sharir, M. (1992). Repeated angles in the plane and related problems. \textit{Journal of Combinatorial Theory, Series A}, 59(1), 12–22.
\end{thebibliography}
\end{solution}

\begin{problem}{User prompt (Erd\H{o}s-654)}
Let $x_1,\ldots,x_n\in \mathbb{R}^2$ with no four points on a circle. Must there exist some $x_i$ with at least $(1-o(1))n$ distinct distances to other $x_i$?
\end{problem}

\begin{solution}{Model Response (\emph{Aletheia})}
    Let $S$ be a set of $n$ points in the Euclidean plane $\mathbb{R}^2$. The set $S$ is said to have \textit{no four points on a circle} if the intersection of $S$ with any Euclidean circle contains at most three points. The problem asks whether this condition forces the existence of at least one point $x \in S$ such that the number of distinct distances from $x$ to $S \setminus \{x\}$, denoted $|\mathcal{D}(x)|$, satisfies $|\mathcal{D}(x)| \ge (1-o(1))n$.

We prove that this assertion is false. We construct a family of point sets where every point determines at most roughly $\frac{3}{4}n$ distinct distances.

\section{Construction}

Let $m \ge 10$ be an integer, and let $n = 4m$. Define the index set $K = \{10, 11, \dots, m+9\}$.
We define two sets of points $P$ and $Q$ located on the $y$-axis and $x$-axis, respectively:
\[
P = \left\{ (0, y) \in \mathbb{R}^2 \;\middle|\; y \in \{ 3^k, -3^k : k \in K \} \right\},
\]
\[
Q = \left\{ (x, 0) \in \mathbb{R}^2 \;\middle|\; x \in \{ 2^j, -2^j : j \in K \} \right\}.
\]
Let $S = P \cup Q$.
Since all coordinates are non-zero (powers of primes are positive), neither set contains the origin. Thus $P \cap Q = \emptyset$.
The size of each subset is $2|K| = 2m$. The total size of $S$ is $4m = n$.

\section{Geometric Verification}

We interpret ``circle'' in the standard Euclidean sense (a locus of points equidistant from a center). Note that in $\mathbb{R}^2$, a straight line is not a circle.

\begin{theorem}
No four points of $S$ lie on a circle.
\end{theorem}

\begin{proof}
A Euclidean circle intersects a straight line in at most two points. Since $P$ lies on the line $x=0$ and $Q$ lies on the line $y=0$, any circle can contain at most two points from $P$ and at most two points from $Q$.
Consequently, for a circle to contain four points of $S$, it must contain exactly two points from $P$ and exactly two points from $Q$.

Suppose such a circle $\mathcal{C}$ exists. Let the points in $P \cap \mathcal{C}$ be $(0, y_1)$ and $(0, y_2)$, and the points in $Q \cap \mathcal{C}$ be $(x_1, 0)$ and $(x_2, 0)$.
The chords formed by these pairs intersect at the origin $(0,0)$. By the Power of a Point Theorem, the product of the signed lengths of the segments from the intersection point must be equal. In terms of coordinates, this implies:
\[
y_1 y_2 = x_1 x_2.
\]
Taking absolute values yields:
\[
|y_1| |y_2| = |x_1| |x_2|.
\]
By construction, $|y_i| = 3^{k_i}$ and $|x_i| = 2^{j_i}$ for some $k_i, j_i \in K$. Substituting these forms:
\[
3^{k_1} \cdot 3^{k_2} = 2^{j_1} \cdot 2^{j_2} \implies 3^{k_1 + k_2} = 2^{j_1 + j_2}.
\]
By the Fundamental Theorem of Arithmetic, a power of 3 equals a power of 2 if and only if both exponents are zero. However, since $k, j \ge 10$, the sums of exponents satisfy $k_1+k_2 \ge 20$ and $j_1+j_2 \ge 20$. Thus, equality is impossible.

This contradiction implies that no circle passes through two points of $P$ and two points of $Q$. Since any other combination of four points involves at least three collinear points (which cannot lie on a circle), we conclude that no four points of $S$ lie on a circle.
\end{proof}

\section{Upper Bound on Distinct Distances}

We establish that every point determines fewer than $\frac{3}{4}n$ distinct distances.

\begin{lemma}
For any $u \in S$, distances from $u$ to points on the same axis are integers, while distances to points on the orthogonal axis are irrational. Thus, the set of distances to $P$ and the set of distances to $Q$ are disjoint.
\end{lemma}

\begin{proof}
We consider two cases based on the location of $u$.

\paragraph{Case 1: $u \in P$.}
Let $u = (0, Y)$ with $|Y| = 3^{k_0}$ ($k_0 \ge 10$).
Distances to other points $v \in P$ correspond to distances between $(0, Y)$ and $(0, y)$, which are $|Y-y|$. Since coordinates are integers, these distances are integers.
Distances to points $w = (x, 0) \in Q$ are given by $\sqrt{Y^2 + x^2}$.
Suppose such a distance is an integer $z$. Then $z^2 - Y^2 = x^2$, which implies:
\[
z^2 - 3^{2k_0} = 2^{2j}.
\]
Factoring the difference of squares, $(z-3^{k_0})(z+3^{k_0}) = 2^{2j}$. The factors must be powers of 2, say $2^a$ and $2^b$, with $a < b$. The difference between factors is:
\[
(z+3^{k_0}) - (z-3^{k_0}) = 2 \cdot 3^{k_0} = 2^b - 2^a = 2^a(2^{b-a} - 1).
\]
Since $3^{k_0}$ is odd, comparing the powers of 2 gives $2^a = 2 \implies a=1$.
The remaining factor yields $3^{k_0} = 2^{b-1} - 1$.
Taking modulo 3: $0 \equiv 2^{b-1} - 1 \pmod 3$, implying $2^{b-1} \equiv 1 \pmod 3$. Thus $b-1$ is even; let $b-1 = 2t$.
Then $3^{k_0} = 2^{2t} - 1 = (2^t - 1)(2^t + 1)$.
Since the product is a power of 3, both factors are powers of 3. Their difference is $(2^t+1) - (2^t-1) = 2$.
The only powers of 3 differing by 2 are 1 and 3. Thus $2^t-1 = 1 \implies t=1$.
Substituting $t=1$ gives $3^{k_0} = 3 \implies k_0 = 1$.
However, by construction $k_0 \ge 10$. Thus, no integer solution exists.

\paragraph{Case 2: $u \in Q$.}
Let $u = (X, 0)$ with $|X| = 2^{j_0}$ ($j_0 \ge 10$).
Distances to points in $Q$ are integers $|X-x|$.
Distances to points in $P$ are $\sqrt{X^2 + y^2}$.
Assume an integer distance $z$. Then $z^2 - 2^{2j_0} = 3^{2k}$.
Factorizing: $(z-2^{j_0})(z+2^{j_0}) = 3^{2k}$.
Let factors be $3^a$ and $3^b$ ($a < b$). Difference:
\[
2 \cdot 2^{j_0} = 3^b - 3^a = 3^a(3^{b-a}-1).
\]
Since 2 is coprime to 3, $a=0$.
Then $2^{j_0+1} = 3^b - 1$.
Consider modulo 4: $0 \equiv 3^b - 1 \implies 3^b \equiv 1 \pmod 4$. Thus $b$ is even; let $b=2s$.
Then $2^{j_0+1} = (3^s-1)(3^s+1)$.
The factors are powers of 2 differing by 2.
Thus $3^s-1 = 2 \implies s=1$.
This implies $2^{j_0+1} = 3^2-1 = 8$, so $j_0+1=3 \implies j_0=2$.
But $j_0 \ge 10$. Thus, no integer solution exists.

In both cases, distances to the orthogonal axis are purely irrational, while distances along the resident axis are integers. Thus, the sets are disjoint.
\end{proof}

\begin{theorem}
For every point $u \in S$, $|\mathcal{D}(u)| < \frac{3}{4}n$.
\end{theorem}

\begin{proof}
Assume without loss of generality that $u \in P$.
By the Lemma, $|\mathcal{D}(u)| = |\mathcal{D}_P| + |\mathcal{D}_Q|$.
\begin{enumerate}
    \item \textbf{Bound for $\mathcal{D}_Q$:}
    The distances are $\{\sqrt{Y^2 + 4^j} \mid j \in K\}$. Since the function $f(j) = Y^2 + 4^j$ is strictly monotonic, there are exactly $|K| = m$ distinct values.
    \[ |\mathcal{D}_Q| = m. \]
    \item \textbf{Bound for $\mathcal{D}_P$:}
    Since $u \in P$, there are $|P|-1 = 2m-1$ other points in $P$. The number of distinct distances cannot exceed the number of target points.
    \[ |\mathcal{D}_P| \le 2m-1. \]
\end{enumerate}
Combining these:
\[
|\mathcal{D}(u)| \le m + (2m-1) = 3m - 1.
\]
Recalling $n = 4m$, we have $3m = \frac{3}{4}n$. Thus:
\[
|\mathcal{D}(u)| \le \frac{3}{4}n - 1 < \frac{3}{4}n.
\]
By symmetry, the same bound applies if $u \in Q$.
\end{proof}

\section{Conclusion}

Since $\frac{3}{4}n < (1-\epsilon)n$ for any $\epsilon \in (0, 0.25)$, our construction demonstrates that a set with no four points on a circle need not contain any point with $(1-o(1))n$ distinct distances.

\begin{remark}[General Position]
The set $S$ contains collinear points. However, if one requires the set to be in \textit{general position} (no three collinear), the answer remains negative. Dumitrescu \cite{Dumitrescu2008} established the existence of sets of $N$ points in general position with no four concyclic points such that the sum of distinct distances determined by all points is $O(N^2 / \sqrt{\log N})$. By the pigeonhole principle, or by extracting a suitable subset, one can find a set satisfying these strict conditions where \textit{every} point determines $o(N)$ distinct distances. While our explicit construction suffices to disprove the claim in the problem statement, this result illustrates that the non-existence holds even under stronger constraints.
\end{remark}

\begin{thebibliography}{9}
\bibitem{Dumitrescu2008} A. Dumitrescu, ``Planar point sets with forbidden 4-point patterns and few distinct distances,'' \textit{Discrete Mathematics}, vol. 308, no. 23, pp. 5598--5603, 2008.
\bibitem{Mihailescu2004} P. Mih\u{a}ilescu, ``Primary Cyclotomic Units and a Proof of Catalan's Conjecture,'' \textit{Journal f\"{u}r die reine und angewandte Mathematik}, vol. 572, pp. 167--195, 2004.
\end{thebibliography}
\end{solution}

\begin{problem}{User prompt (Erd\H{o}s-659)}
    Is there a set of $n$ points in $\mathbb{R}^2$ such that every subset of $4$ points determines at least $3$ distances, yet the total number of distinct distances is\[\ll \frac{n}{\sqrt{\log n}}?\]
\end{problem}

\begin{solution}{Model Response (\emph{Aletheia})}
\begin{theorem}
There exists a set $P$ of $n$ points in the Euclidean plane $\mathbb{R}^2$ such that every subset of $4$ points determines at least $3$ distinct distances, and the total number of distinct distances determined by $P$ is $O(n/\sqrt{\log n})$.
\end{theorem}

Consequently, the answer to the question is \textbf{YES} (interpreting the notation $\ll$ as the asymptotic bound $O(\cdot)$).

\begin{proof}
We construct the set $P$ as a subset of a specific integer lattice in $\mathbb{R}^2$. This lattice is chosen to ensure that all squared distances are integers (ruling out irrational distance ratios) and that it possesses no geometric symmetries that would allow for a 4-point subset with fewer than 3 distances.

\subsubsection*{1. Construction of the Lattice and Point Set}

Let $K = \mathbb{Q}(\sqrt{-7})$. The ring of integers $\mathcal{O}_K$ is generated over $\mathbb{Z}$ by $1$ and $\omega = \frac{1 + i\sqrt{7}}{2}$. We embed $\mathcal{O}_K$ into $\mathbb{R}^2$ to form the lattice $\Lambda$:
\[ \Lambda = \left\{ u = m(1, 0) + k\left(\frac{1}{2}, \frac{\sqrt{7}}{2}\right) \;\middle|\; m, k \in \mathbb{Z} \right\}. \]
The squared Euclidean norm of a vector $u = m + k\omega$ is given by the field norm $N(u)$:
\[ \|u\|^2 = \left(m + \frac{k}{2}\right)^2 + \left(\frac{k\sqrt{7}}{2}\right)^2 = m^2 + mk + 2k^2. \]
Let $Q(m, k) = m^2 + mk + 2k^2$. Since $m, k \in \mathbb{Z}$, the squared distance between any two points in $\Lambda$ is a non-negative integer. Thus, the ratio of any two squared distances in $\Lambda$ must be rational.

Let $P_n$ be the subset of $\Lambda$ consisting of the $n$ points closest to the origin. Since $\Lambda$ is a lattice, these points are contained within a disk of radius $R$, where $n \approx \frac{\pi R^2}{\text{vol}(\Lambda)}$. Thus $R^2 = \Theta(n)$.

\subsubsection*{2. Bound on Distinct Distances}

The set of distinct distances determined by $P_n$ is a subset of $\{\sqrt{N} \mid N \in \mathbb{Z}_{\ge 0}, N \le 4R^2, N \text{ represented by } Q\}$.
The quadratic form $Q(m, k) = m^2 + mk + 2k^2$ is positive definite with discriminant $\Delta = 1^2 - 4(2) = -7$.
A theorem by Bernays (1912), generalizing the Landau-Ramanujan theorem, states that the number of positive integers $h \le X$ representable by a positive definite binary quadratic form is asymptotically
\[ \mathcal{B}(X) \sim C \frac{X}{\sqrt{\log X}}, \]
for some constant $C > 0$. Since the maximum squared distance in $P_n$ is $X = 4R^2 = O(n)$, the number of distinct distances satisfies:
\[ |D(P_n)| \le \mathcal{B}(O(n)) = O\left(\frac{n}{\sqrt{\log n}}\right). \]

\subsubsection*{3. Verification of the 4-Point Condition}

We claim that every subset of 4 distinct points in $\Lambda$ determines at least 3 distinct distances.
Since no set of 4 points in $\mathbb{R}^2$ can determine exactly 1 distance, it suffices to prove that $\Lambda$ contains no subset of 4 points determining exactly 2 distinct distances.

According to the classification of 4-point sets with exactly 2 distinct distances, any such set must be similar to one of the following configurations:
\begin{enumerate}
    \item \textbf{Square} (Squared distances ratio 2).
    \item \textbf{Rhombus} composed of two equilateral triangles (Squared distances ratio 3).
    \item \textbf{Isosceles Trapezoid} composed of three equilateral triangles (Squared distances ratio 3).
    \item \textbf{Kite} composed of two equilateral triangles (Squared distances ratio 3).
    \item \textbf{Equilateral Triangle with Centroid} (Squared distances ratio 3).
    \item \textbf{Regular Pentagon Subsets} (Squared distances ratios involving $\sqrt{5}$).
\end{enumerate}

Observe that configurations 2 through 5 all contain an equilateral triangle as a sub-configuration. Configuration 6 involves irrational squared distance ratios (specifically $\frac{3+\sqrt{5}}{2}$), which cannot exist in $\Lambda$.

\begin{lemma}
The lattice $\Lambda$ contains no equilateral triangles.
\end{lemma}
\begin{proof}
Suppose points $0, u, v \in \Lambda$ form an equilateral triangle. Then $\|u\|^2 = \|v\|^2 = \|u-v\|^2 = L$.
Using complex numbers corresponding to the lattice points, $v/u = \frac{1 \pm i\sqrt{3}}{2}$.
For $u, v \in \Lambda$, the quotient $v/u$ must lie in the field of fractions of $\mathcal{O}_K$, which is $K = \mathbb{Q}(\sqrt{-7})$.
However, $\sqrt{3} \notin \mathbb{Q}(\sqrt{-7})$. Alternatively, equating the area $\frac{\sqrt{3}}{4}L$ to the lattice area $\frac{k\sqrt{7}}{4}$ yields $\sqrt{3/7} \in \mathbb{Q}$, which is false.
\end{proof}

\begin{lemma}
The lattice $\Lambda$ contains no squares.
\end{lemma}
\begin{proof}
Suppose points $0, u, w, u+w \in \Lambda$ form a square. In complex numbers, $w/u = \pm i$.
This requires $i \in K = \mathbb{Q}(\sqrt{-7})$. Let $i = a + b\sqrt{-7}$ with $a, b \in \mathbb{Q}$.
Squaring yields $-1 = a^2 - 7b^2 + 2ab\sqrt{-7}$. Thus $2ab = 0$.
If $b=0$, $a^2 = -1$ (impossible). If $a=0$, $7b^2 = 1$ (impossible). Thus $i \notin K$.
\end{proof}

\subsubsection*{Conclusion}
The set $P_n \subset \Lambda$ satisfies both requirements:
\begin{enumerate}
    \item Every subset of 4 points determines $\ge 3$ distinct distances.
    \item The total number of distinct distances is $O(n/\sqrt{\log n})$.
\end{enumerate}
\end{proof}

\textbf{Answer: YES}
\end{solution}

\begin{problem}{User Prompt (Erd\H{o}s-705)}
Let $G$ be a finite unit distance graph in $\mathbb{R}^2$ (i.e. the vertices are a finite collection of points in $\mathbb{R}^2$ and there is an edge between two points if and only if the distance between them is $1$). Is there some $k$ such that if $G$ has girth $\geq k$ (i.e. $G$ contains no cycles of length $<k$) then $\chi(G)\leq 3$?
\end{problem}

\begin{solution}{Model Response (\emph{Aletheia})}

\noindent The answer to the question is \textbf{No}.

\begin{theorem}
There does not exist an integer $k$ such that every finite unit distance graph $G$ in $\mathbb{R}^2$ with girth at least $k$ satisfies $\chi(G) \leq 3$.
\end{theorem}

\begin{proof}
We proceed by contradiction. Suppose there exists an integer $k$ such that for every finite unit distance graph $G$ in the Euclidean plane, the condition $\text{girth}(G) \ge k$ implies $\chi(G) \leq 3$.

Recall that a unit distance graph $G = (V, E)$ is defined by a finite set of vertices $V \subset \mathbb{R}^2$ where $\{u, v\} \in E$ if and only if $\|u - v\| = 1$.

We invoke a result established by Paul O'Donnell in 2000, which generalizes earlier constructions of 4-chromatic unit distance graphs by Moser (girth 3) and Wormald (girth 5).

\paragraph{Theorem (O'Donnell, 2000):} For every integer $g \ge 3$, there exists a finite unit distance graph $H_g$ in $\mathbb{R}^2$ such that:
\begin{enumerate}
    \item The girth of $H_g$ is greater than $g$.
    \item The chromatic number of $H_g$ is 4.
\end{enumerate}

Let $g = k$. By O'Donnell's theorem, there exists a finite unit distance graph $H_k$ with $\text{girth}(H_k) > k$. Since the girth is strictly greater than $k$, the condition $\text{girth}(H_k) \ge k$ holds. Therefore, under our initial assumption, $H_k$ must satisfy $\chi(H_k) \leq 3$.

However, O'Donnell's construction guarantees that $\chi(H_k) = 4$. This yields the contradiction $4 \leq 3$.

Thus, the assumption is false. No such integer $k$ exists because there are finite unit distance graphs of arbitrarily large girth that require 4 colors.
\end{proof}

\section*{Discussion}
The problem of determining whether girth constraints can bound the chromatic number of unit distance graphs has been extensively studied in geometric graph theory.



\begin{enumerate}
    \item \textbf{Small Girth Examples:} The classical \textbf{Moser spindle} (1961) is a unit distance graph with 7 vertices and 11 edges that requires 4 colors. It contains 3-cycles, so its girth is 3.
    \item \textbf{Girth 5:} Nicholas Wormald (1979) constructed a unit distance graph with 6448 vertices that is triangle-free and square-free (girth 5) but still requires 4 colors. This demonstrated that excluding $C_3$ and $C_4$ is insufficient to force 3-colorability.
    \item \textbf{Arbitrary Girth:} Paul O'Donnell's work (2000) completed the picture by showing that for \textit{any} fixed cycle length exclusion, one can still construct a 4-chromatic unit distance graph. His construction involves defining a sequence of abstract graphs with high girth and chromatic number 4, then carefully embedding them into $\mathbb{R}^2$ such that edges map to unit distances and no spurious unit distances (which could form short cycles) are introduced.
\end{enumerate}

Consequently, the class of unit distance graphs behaves similarly to general graphs in this respect: high girth does not imply low chromatic number (specifically, it does not imply $\chi \le 3$). Note that the Hadwiger-Nelson problem remains open regarding whether a unit distance graph with $\chi \ge 5$ exists (recently, de Grey proved $\chi \ge 5$ for the infinite plane graph, and finite subgraphs with $\chi=5$ have been found, but their girth properties are separate questions). However, for the specific bound of 3, O'Donnell's result provides a definitive negative answer.

\section*{References}
\begin{enumerate}
    \item \textbf{O'Donnell, Paul.} ``Arbitrary Girth, 4-Chromatic Unit Distance Graphs in the Plane I: Graph Description.'' \textit{Geombinatorics} 9, no. 3 (2000): 145–152.
    \item \textbf{O'Donnell, Paul.} ``Arbitrary Girth, 4-Chromatic Unit Distance Graphs in the Plane II: Graph Embedding.'' \textit{Geombinatorics} 9, no. 4 (2000): 180–193.
    \item \textbf{Wormald, Nicholas C.} ``A 4-chromatic graph with a special plane drawing.'' \textit{Journal of the Australian Mathematical Society (Series A)} 28, no. 1 (1979): 1–8.
    \item \textbf{Graham, Ronald L.} ``Euclidean Ramsey Theory.'' In \textit{Handbook of Discrete and Computational Geometry}, edited by J.E. Goodman and J. O'Rourke, Chapter 11. CRC Press.
\end{enumerate}
\end{solution}

\begin{problem}{User Prompt (Erd\H{o}s-935)}
For any integer $n=\prod p^{k_p}$ let $Q_2(n)$ be the powerful part of $n$, so that\[Q_2(n) = \prod_{\substack{p\\ k_p\geq 2}}p^{k_p}.\]Is it true that, for every $\epsilon>0$ and $\ell\geq 1$, if $n$ is sufficiently large then\[Q_2(n(n+1)\cdots(n+\ell))<n^{2+\epsilon}?\]If $\ell\geq 2$ then is\[\limsup_{n\to \infty}\frac{Q_2(n(n+1)\cdots(n+\ell))}{n^2}\]infinite? If $\ell\geq 2$ then is\[\lim_{n\to \infty}\frac{Q_2(n(n+1)\cdots(n+\ell))}{n^{\ell+1}}=0?\]
\end{problem}

\begin{solution}{Model Response (\emph{Aletheia})}
   We provide a complete and rigorous resolution to all three parts of the problem.
\begin{enumerate}
    \item \textbf{Upper Bound:} We answer \textbf{Yes}. We invoke the theorem of J. Cilleruelo (2008), which establishes that the limit superior of the log-ratio of the powerful part of a product of consecutive integers to $\log n$ is exactly 2. Consequently, for any $\epsilon > 0$, the powerful part is bounded by $n^{2+\epsilon}$ for sufficiently large $n$.
    \item \textbf{Infinite Limit Superior:} We answer \textbf{Yes}. We demonstrate that for $\ell \ge 2$, the ratio $Q_2(P(n, \ell))/n^2$ is unbounded. By constructing a sequence of integers $\{n_k\}$ based on the solutions to the Pell equation $x^2 - 8y^2 = 1$, we ensure that $n_k$ and $n_k+1$ are powerful numbers. We then use properties of the fundamental unit in $\mathbb{Z}[\sqrt{8}]$ to show that $n_k+2$ is divisible by the square of arbitrarily large primes $p \equiv 5 \pmod 8$, thereby making the powerful part arbitrarily large relative to $n^2$.
    \item \textbf{Vanishing Limit:} We answer \textbf{Yes}. This follows directly from the upper bound established in the first part. Since the powerful part grows at most as $n^{2+\epsilon}$ and the denominator grows as $n^{\ell+1}$ (with $\ell+1 \ge 3$), the ratio converges to zero.
\end{enumerate}


\section{Introduction}

Let $n$ be a positive integer. The \textit{powerful part} of $n$, denoted by $\Qtwo(n)$, is the maximal divisor of $n$ such that every prime factor of the divisor appears with multiplicity at least 2. Equivalently, if the prime factorization of $n$ is $n = \prod_{p} p^{v_p(n)}$, then
\[ \Qtwo(n) = \prod_{\substack{p \mid n \\ v_p(n) \ge 2}} p^{v_p(n)}. \]
Let $\ell \ge 1$ be a fixed integer. We examine the product of $\ell+1$ consecutive integers starting at $n$:
\[ P(n, \ell) = \prod_{i=0}^\ell (n+i). \]
We address three specific questions concerning the asymptotic growth of $\Qtwo(P(n, \ell))$ as $n \to \infty$.

\section{Upper Bound on the Powerful Part}

\textbf{Question 1.} Is it true that for every $\epsilon > 0$ and $\ell \ge 1$, if $n$ is sufficiently large, then $\Qtwo(P(n, \ell)) < n^{2+\epsilon}$?

\begin{theorem}
Let $\ell \ge 1$ and $\epsilon > 0$. There exists an integer $N$ such that for all $n > N$,
\[ \Qtwo(P(n, \ell)) < n^{2+\epsilon}. \]
\end{theorem}

\begin{proof}
The product $P(n, \ell)$ involves $k = \ell + 1$ consecutive integers, where $k \ge 2$. The logarithmic order of the powerful part for such products was precisely determined by Cilleruelo \cite{Cilleruelo2008}.

\begin{theorem}[Cilleruelo, 2008]
For any integer $k \ge 2$, let $P(n, k-1)$ denote the product of $k$ consecutive integers starting at $n$. Then
\[ \limsup_{n \to \infty} \frac{\log \Qtwo(P(n, k-1))}{\log n} = 2. \]
\end{theorem}

Using this result with $k = \ell + 1$, we have:
\[ \limsup_{n \to \infty} \frac{\log \Qtwo(P(n, \ell))}{\log n} = 2. \]
By the definition of the limit superior, for any preassigned $\epsilon > 0$, there exists a threshold $N$ such that for all $n > N$:
\[ \frac{\log \Qtwo(P(n, \ell))}{\log n} < 2 + \epsilon. \]
Since $\log n > 0$ for $n > N$ (assuming $N \ge 2$), multiplying by $\log n$ yields:
\[ \log \Qtwo(P(n, \ell)) < (2+\epsilon) \log n. \]
Exponentiating both sides gives:
\[ \Qtwo(P(n, \ell)) < n^{2+\epsilon}. \]
This establishes the upper bound for all sufficiently large $n$.
\end{proof}

\section{Infinite Limit Superior}

\textbf{Question 2.} If $\ell \ge 2$, is $\limsup_{n\to \infty}\frac{\Qtwo(P(n, \ell))}{n^2}$ infinite?

\begin{theorem}
For any fixed integer $\ell \ge 2$,
\[ \limsup_{n \to \infty} \frac{\Qtwo(P(n, \ell))}{n^2} = \infty. \]
\end{theorem}

\begin{proof}
Assume $\ell \ge 2$. The product $P(n, 2) = n(n+1)(n+2)$ is a divisor of $P(n, \ell)$. By the property that $A \mid B \implies \Qtwo(A) \le \Qtwo(B)$, we have:
\[ \frac{\Qtwo(P(n, \ell))}{n^2} \ge \frac{\Qtwo(P(n, 2))}{n^2}. \]
Thus, proving the divergence for $\ell=2$ is sufficient. We construct a specific sequence of integers $\{n_k\}$ along which the normalized powerful part is unbounded.

Consider the Pell equation:
\begin{equation} \label{eq:pell}
x^2 - 8y^2 = 1.
\end{equation}
Let $\alpha = 3 + \sqrt{8}$ be the fundamental unit of $\Z[\sqrt{8}]$. Define the sequences $(x_k)_{k \ge 1}$ and $(y_k)_{k \ge 1}$ of positive integers by the relation:
\[ x_k + y_k\sqrt{8} = \alpha^k. \]
We set $n_k = 8y_k^2$. From (\ref{eq:pell}), we derive $n_k+1 = x_k^2$ and $n_k+2 = x_k^2+1$.

\textbf{Analysis of Powerful Properties:}
\begin{enumerate}
\item \textbf{Term $n_k$:} We have $n_k = 2^3 y_k^2$. Every prime factor of $y_k^2$ appears with an even multiplicity. The prime $2$ has multiplicity $3 + 2v_2(y_k) \ge 3$. Thus, every prime factor of $n_k$ appears with multiplicity at least 2. Therefore, $n_k$ is a powerful number, so $\Qtwo(n_k) = n_k$. Note that $v_2(n_k)$ is odd.
\item \textbf{Term $n_k+1$:} Since $n_k+1 = x_k^2$, it is a perfect square and hence powerful. Thus $\Qtwo(n_k+1) = n_k+1$. Moreover, $x_k^2 - 8y_k^2 = 1$ implies $x_k$ is odd, so $n_k+1$ is odd.
\item \textbf{Term $n_k+2$:} As $x_k$ is odd, $x_k^2 \equiv 1 \pmod 8$. Thus $n_k+2 = x_k^2+1 \equiv 2 \pmod 8$. Consequently, $v_2(n_k+2) = 1$.
\end{enumerate}

\textbf{Calculating $\Qtwo(P(n_k, 2))$:}
Let $P_k = n_k(n_k+1)(n_k+2)$.
We examine the pairwise greatest common divisors:
\begin{itemize}
\item $\gcd(n_k, n_k+1) = 1$.
\item $\gcd(n_k+1, n_k+2) = 1$.
\item $\gcd(n_k, n_k+2) = \gcd(8y_k^2, x_k^2+1) = 2$.
\end{itemize}
The only shared prime factor is 2. The total valuation of 2 is:
\[ v_2(P_k) = v_2(n_k) + v_2(n_k+1) + v_2(n_k+2) = v_2(n_k) + 0 + 1. \]
Since $v_2(n_k)$ is odd, the sum is even. Thus, the factor $2^{v_2(P_k)}$ is powerful.
Because odd prime factors are disjoint across the three terms, the powerful part function acts multiplicatively on the odd components.
\[ \Qtwo(P_k) = 2^{v_2(P_k)} \cdot \Qtwo(\text{odd}(n_k)) \cdot \Qtwo(n_k+1) \cdot \Qtwo(\text{odd}(n_k+2)). \]
Since $n_k$ is powerful, $\Qtwo(\text{odd}(n_k)) = n_k / 2^{v_2(n_k)}$.
Since $v_2(n_k+2)=1$, the odd part of $n_k+2$ is $(n_k+2)/2$, and $\Qtwo((n_k+2)/2) = \Qtwo(n_k+2)$.
Substituting these into the expression:
\begin{align*}
\Qtwo(P_k) &= 2^{v_2(n_k)+1} \cdot \frac{n_k}{2^{v_2(n_k)}} \cdot (n_k+1) \cdot \Qtwo(n_k+2) \\
&= 2 \cdot n_k \cdot (n_k+1) \cdot \Qtwo(n_k+2).
\end{align*}
Forming the ratio of interest:
\[ \frac{\Qtwo(P_k)}{n_k^2} = 2 \left( 1 + \frac{1}{n_k} \right) \Qtwo(n_k+2). \]
This ratio exceeds $2 \Qtwo(n_k+2)$. It remains to prove that $\Qtwo(n_k+2)$ is unbounded.

\begin{lemma}
Let $p$ be any prime with $p \equiv 5 \pmod 8$. There exists an index $k \in \N$ such that $n_k + 2 \equiv 0 \pmod{p^2}$.
\end{lemma}
\begin{proof}
Recall $n_k+2 = x_k^2+1$. We seek $k$ such that $x_k^2 \equiv -1 \pmod{p^2}$.
Since $p \equiv 5 \pmod 8$, the Legendre symbol is $(2/p) = -1$. Thus $p$ is inert in $\Z[\sqrt{2}]$.
Let $\alpha = 3 + \sqrt{8} = (1+\sqrt{2})^2$.
In $\F_{p^2} \cong \Z[\sqrt{2}]/(p)$, the Frobenius map $\sigma(z) = z^p$ fixes elements of $\F_p$ and sends $\sqrt{2} \to -\sqrt{2}$.
Therefore,
\[ (1+\sqrt{2})^p = 1 - \sqrt{2}. \]
Then
\[ \alpha^{(p+1)/2} = \left((1+\sqrt{2})^2\right)^{(p+1)/2} = (1+\sqrt{2})^{p+1} = (1+\sqrt{2})(1-\sqrt{2}) = 1 - 2 = -1. \]
Let $m = (p+1)/2$. Since $p \equiv 5 \pmod 8$, $m$ is an odd integer.
Thus $\alpha^m \equiv -1 \pmod p$.
Next, we lift this congruence to modulo $p^2$. Write $\alpha^m = -1 + p\delta$ for some $\delta \in \Z[\sqrt{8}]$.
Raising to the power $p$ (which is odd):
\[ \alpha^{mp} = (-1 + p\delta)^p = \sum_{j=0}^p \binom{p}{j}(-1)^{p-j}(p\delta)^j. \]
Since $p \ge 3$, terms with $j \ge 1$ are divisible by $p^2$. The term for $j=0$ is $(-1)^p = -1$.
Thus, $\alpha^{mp} \equiv -1 \pmod{p^2}$.
Let $M = mp$. Since $m$ and $p$ are odd, $M$ is odd.
Define $k$ by $2k = M+1$. Note that $k$ is an integer.
Then
\[ \alpha^{2k} = \alpha^{M+1} = \alpha^M \cdot \alpha \equiv (-1) \cdot \alpha = -\alpha \pmod{p^2}. \]
Since $\alpha$ is a unit, $\alpha^{-2k} \equiv -\alpha^{-1} \pmod{p^2}$.
From the recurrence relation $2x_k = \alpha^k + \alpha^{-k}$, squaring yields:
\[ 4x_k^2 = \alpha^{2k} + \alpha^{-2k} + 2. \]
Substituting the modular expressions:
\[ 4x_k^2 \equiv -\alpha - \alpha^{-1} + 2 \pmod{p^2}. \]
Since $\alpha + \alpha^{-1} = (3+\sqrt{8}) + (3-\sqrt{8}) = 6$, we have:
\[ 4x_k^2 \equiv -6 + 2 = -4 \pmod{p^2}. \]
Since $p$ is odd, 4 is a unit modulo $p^2$. Dividing by 4 gives:
\[ x_k^2 \equiv -1 \pmod{p^2}. \]
Thus $n_k+2 = x_k^2+1 \equiv 0 \pmod{p^2}$.
\end{proof}

\textbf{Conclusion of Divergence:}
Let $B > 0$ be an arbitrary constant. By Dirichlet's theorem on primes in arithmetic progressions, there exist infinitely many primes $p \equiv 5 \pmod 8$. Select such a prime satisfying $p^2 > B/2$.
By the Lemma, there exists an index $k$ such that $p^2 \mid (n_k+2)$.
Since $p^2$ is a perfect square, it is a powerful number. Divisors of powerful numbers are not necessarily powerful, but $p^2$ itself satisfies the powerful condition. Thus $p^2$ must divide $\Qtwo(n_k+2)$.
Therefore, $\Qtwo(n_k+2) \ge p^2$.
Returning to the ratio:
\[ \frac{\Qtwo(P_k)}{n_k^2} > 2 \Qtwo(n_k+2) \ge 2p^2 > B. \]
Since $B$ can be arbitrarily large, we conclude:
\[ \limsup_{n \to \infty} \frac{\Qtwo(P(n, \ell))}{n^2} = \infty. \]
\end{proof}

\section{Convergence to Zero}

\textbf{Question 3.} If $\ell \ge 2$, is $\lim_{n \to \infty} \frac{\Qtwo(P(n, \ell))}{n^{\ell+1}} = 0$?

\begin{theorem}
For any integer $\ell \ge 2$,
\[ \lim_{n \to \infty} \frac{\Qtwo(P(n, \ell))}{n^{\ell+1}} = 0. \]
\end{theorem}

\begin{proof}
Let $\ell \ge 2$. By Theorem 1, we know that for any $\epsilon > 0$, there is a constant $N$ such that whenever $n > N$:
\[ 0 \le \Qtwo(P(n, \ell)) < n^{2+\epsilon}. \]
Let us evaluate the ratio $A_n = \frac{\Qtwo(P(n, \ell))}{n^{\ell+1}}$.
Substituting the upper bound for $n > N$:
\[ 0 \le A_n < \frac{n^{2+\epsilon}}{n^{\ell+1}} = n^{1+\epsilon - \ell}. \]
Because $\ell \ge 2$, we have $1 - \ell \le -1$.
Choosing $\epsilon = 0.5$, the exponent becomes:
\[ 1 + 0.5 - \ell = 1.5 - \ell \le 1.5 - 2 = -0.5. \]
Thus, for $n > N$:
\[ 0 \le A_n < n^{-0.5} = \frac{1}{\sqrt{n}}. \]
Since $\lim_{n \to \infty} n^{-0.5} = 0$, by the Squeeze Theorem:
\[ \lim_{n \to \infty} A_n = 0. \]
Therefore, the limit is zero.
\end{proof}

\begin{thebibliography}{9}

\bibitem{Cilleruelo2008}
J. Cilleruelo, \textit{The powerful part of the product of consecutive integers}, Bull. London Math. Soc. \textbf{40} (2008), 873--877.

\end{thebibliography}
\end{solution}

\begin{problem}{User Prompt (Erd\H{o}s-992)}
Let $x_1<x_2<\cdots$ be an infinite sequence of integers. Is it true that, for almost all $\alpha \in [0,1]$, the discrepancy\[D(N)=\max_{I\subseteq [0,1]} \lvert \#\{ n\leq N : \{ \alpha x_n\}\in I\} - \lvert I\rvert N\rvert\]satisfies\[D(N) \ll N^{1/2}(\log N)^{o(1)}?\]Or even\[D(N)\ll N^{1/2}(\log\log N)^{O(1)}?\]
\end{problem}

\begin{solution}{Model Response (\emph{Aletheia})}
\begin{theorem}
The answer is \textbf{no}. There exists a strictly increasing sequence of integers $(x_n)_{n \ge 1}$ such that for almost all $\alpha \in [0,1]$, the discrepancy
\[ D(N) = \sup_{I\subseteq [0,1]} \lvert \#\{ n\leq N : \{ \alpha x_n\}\in I\} - \lvert I\rvert N\rvert \]
satisfies
\[ \limsup_{N \to \infty} \frac{D(N)}{\sqrt{N \log N}} > 0. \]
In particular, the proposed bound $D(N) \ll N^{1/2}(\log N)^{o(1)}$ fails for this sequence for almost all $\alpha$. Consequently, the stronger bound $D(N) \ll N^{1/2}(\log\log N)^{O(1)}$ also fails.
\end{theorem}

\begin{proof}
Let $(x_n)_{n \ge 1}$ be a strictly increasing sequence of integers. The discrepancy $D(N)$ defined in the problem statement corresponds to the unnormalized star discrepancy, often denoted in the literature as $N D_N^*(\alpha)$.

We examine whether the bound
\[ D(N) \ll N^{1/2}(\log N)^{o(1)} \]
holds for almost all $\alpha$. If this bound were valid for every strictly increasing integer sequence, it would imply that for any $\varepsilon > 0$,
\[ D(N) = O(N^{1/2}(\log N)^\varepsilon) \quad \text{almost everywhere}. \]

However, in 1994, Istv\'{a}n Berkes and Walter Philipp proved that the logarithmic power in the discrepancy upper bound cannot be made arbitrarily small for general integer sequences.

\begin{lemma}[Berkes \& Philipp [1]]
There exists a strictly increasing sequence of integers $(x_n)_{n \ge 1}$ and a constant $c > 0$ such that for almost all $\alpha \in [0,1]$,
\[ \limsup_{N \to \infty} \frac{D(N)}{\sqrt{N \log N}} \geq c. \]
\end{lemma}

\paragraph{Verification of the Negative Answer:}
Suppose, for contradiction, that $D(N) \ll N^{1/2}(\log N)^{o(1)}$ holds for the sequence identified in the Lemma. This implies that for any $\varepsilon > 0$, there exists a constant $C_\alpha$ such that for all sufficiently large $N$,
\[ D(N) \leq C_\alpha N^{1/2}(\log N)^\varepsilon. \]
Choose $\varepsilon = 1/4$. Then we must have
\[ D(N) \leq C_\alpha N^{1/2}(\log N)^{1/4}. \]
However, the Lemma asserts that there are infinitely many $N$ such that
\[ D(N) \geq c N^{1/2}(\log N)^{1/2}. \]
Combining these inequalities for such $N$ gives
\[ c N^{1/2}(\log N)^{1/2} \leq C_\alpha N^{1/2}(\log N)^{1/4}, \]
which simplifies to
\[ (\log N)^{1/4} \leq \frac{C_\alpha}{c}. \]
This is a contradiction as $N \to \infty$. Thus, the bound $D(N) \ll N^{1/2}(\log N)^{o(1)}$ cannot hold.

Since $\sqrt{N \log N}$ grows strictly faster than $N^{1/2}(\log \log N)^{k}$ for any fixed $k$, the second proposed bound
\[ D(N) \ll N^{1/2}(\log\log N)^{O(1)} \]
is also false for this sequence.
\end{proof}

\begin{remark}
While R. C. Baker [2] established that $D(N) \ll N^{1/2}(\log N)^{3/2+\varepsilon}$ holds almost everywhere for \textit{any} distinct sequence of integers, the gap between the exponents $1/2$ (from the Berkes-Philipp lower bound) and $3/2$ (from Baker's upper bound) indicates that the discrepancy behavior of general integer sequences is significantly worse than that of lacunary sequences (which satisfy the Law of the Iterated Logarithm, $D(N) \sim \sqrt{N \log \log N}$, as shown by Philipp [3]). The Berkes-Philipp counterexample specifically exploits arithmetic structures in the sequence to create correlations that prevent the cancellations necessary for a smaller discrepancy bound.
\end{remark}

\section*{References}

\begin{enumerate}
    \item[{[1]}] I. Berkes and W. Philipp, ``The size of trigonometric and Walsh series and uniform distribution mod 1'', \textit{Journal of the London Mathematical Society} (2) \textbf{50} (1994), 454–464.
    \item[{[2]}] R. C. Baker, ``Metric number theory and the large sieve'', \textit{Journal of the London Mathematical Society} (2) \textbf{24} (1981), 34–40.
    \item[{[3]}] W. Philipp, ``Limit theorems for lacunary series and uniform distribution mod 1'', \textit{Acta Arithmetica} \textbf{26} (1975), 241–251.
\end{enumerate}
\end{solution}


\begin{problem}{User Prompt (Erd\H{o}s-1040)}
    Let $F\subseteq \mathbb{C}$ be a closed infinite set, and let $\mu(F)$ be the infimum of\[\lvert \{ z: \lvert f(z)\rvert < 1\}\rvert,\]as $f$ ranges over all polynomials of the shape $\prod (z-z_i)$ with $z_i\in F$. Is $\mu(F)$ determined by the transfinite diameter of $F$? In particular, is $\mu(F)=0$ whenever the transfinite diameter of $F$ is $\geq 1$?
\end{problem}

\begin{solution}{Model Response (\emph{Aletheia})}
    \section{Introduction}

Let $F \subseteq \mathbb{C}$ be a closed infinite set. We define $\mu(F)$ as the infimum of the area of the lemniscate $\{z \in \mathbb{C} : |P(z)| < 1\}$ taken over all monic polynomials $P$ with roots in $F$.
We address two specific questions regarding the properties of $\mu(F)$:
\begin{enumerate}
    \item Is $\mu(F)$ uniquely determined by the transfinite diameter $d_\infty(F)$?
    \item Does $d_\infty(F) \ge 1$ imply $\mu(F) = 0$?
\end{enumerate}



\section{Part 1: Independence from Transfinite Diameter}

\textbf{Answer: No.}

We prove that $\mu(F)$ is not determined by $d_\infty(F)$. To do so, we exhibit two distinct closed infinite sets, $F_1$ and $F_2$, which have identical transfinite diameters ($d_\infty = 0$) but yield markedly different values for $\mu(F)$.

Recall that for any countable compact set $K$, the transfinite diameter (logarithmic capacity) is zero.

\paragraph{1. Construction of $F_1$ (Positive Area Bound)}

Let $F_1 = \{0\} \cup \left\{ \frac{1}{n} : n \in \mathbb{Z}, n \ge 1 \right\}$.
Since $F_1$ is a countable compact subset of the interval $[0, 1]$, $d_\infty(F_1) = 0$.
Let $P(z) = \prod_{j=1}^m (z - x_j)$ be an arbitrary monic polynomial with roots $x_j \in F_1$.
Since $F_1 \subset [0, 1]$, every root satisfies $x_j \in [0, 1]$.
Consider the disk $D = \{ z : |z - 1/2| < 1/2 \}$.
For any $z \in D$, $|z - 1/2| < 1/2$. Also, since $x_j \in [0, 1]$, $|x_j - 1/2| \le 1/2$.
Applying the triangle inequality:
\[ |z - x_j| \le |z - 1/2| + |1/2 - x_j| < \frac{1}{2} + \frac{1}{2} = 1. \]
Thus, $|P(z)| = \prod |z - x_j| < 1^m = 1$ for all $z \in D$.
Consequently, the lemniscate contains $D$, so:
\[ \text{Area}(\{z : |P(z)| < 1\}) \ge \text{Area}(D) = \frac{\pi}{4}. \]
Taking the infimum over all $P$, we have $\mu(F_1) \ge \pi/4$.

\paragraph{2. Construction of $F_2$ (Vanishing Area)}

Let $R > 4$ be a real parameter. Define
\[ F_2 = \{0, R\} \cup \left\{ \frac{1}{n} : n \ge 1 \right\} \cup \left\{ R + \frac{1}{n} : n \ge 1 \right\}. \]
$F_2$ is a countable compact set, so $d_\infty(F_2) = 0$.
Consider the monic quadratic polynomial $Q(z) = z(z-R)$. Its roots $0, R$ lie in $F_2$.
We compute the area of the lemniscate $L = \{z : |z(z-R)| < 1\}$.
Let $w = z(z-R)$. By completing the square, $w = (z - R/2)^2 - R^2/4$.
Substitute $u = (z - R/2)^2$. The condition $|w| < 1$ transforms to $|u - R^2/4| < 1$.
Solving for $z$: $z = R/2 \pm \sqrt{u}$.
The Jacobian is $|dz/du|^2 = |1/(2\sqrt{u})|^2 = 1/(4|u|)$.
Summing the contributions from both branches of the square root:
\[ \text{Area}(L) = \int_{|u - R^2/4| < 1} \frac{1}{2|u|} dA_u. \]
In the domain of integration $D(R^2/4, 1)$, we have $|u| \ge R^2/4 - 1$. Thus, the integrand is bounded by $\frac{1}{2(R^2/4 - 1)}$.
\[ \text{Area}(L) \le \pi(1)^2 \cdot \frac{1}{2(R^2/4 - 1)} = \frac{2\pi}{R^2 - 4}. \]
As we increase $R$, this area tends to zero. By choosing $R$ sufficiently large (e.g., $R=20$), we ensure $\mu(F_2) < \pi/4$.

\textbf{Conclusion:}
Since $F_1$ and $F_2$ both possess transfinite diameter zero, yet $\mu(F_1) \ge \pi/4$ while $\mu(F_2)$ can be arbitrarily close to zero, $\mu(F)$ is not determined by $d_\infty(F)$.

\section{Part 2: Vanishing Area for Capacity $\ge 1$}

\textbf{Answer: Yes.}

We demonstrate that if $d_\infty(F) \ge 1$, then $\mu(F) = 0$. We divide the proof into two exhaustive cases: unbounded sets and bounded sets.

\paragraph{Case 1: $F$ is Unbounded}

If $F$ is unbounded, it contains pairs of points with arbitrarily large separation.
Fix $a \in F$. Since $F$ is unbounded, there exists a sequence $\{b_k\}_{k \ge 1} \subset F$ such that $R_k = |b_k - a| \to \infty$.
Define the sequence of monic polynomials $P_k(z) = (z - a)(z - b_k)$.
Following the area estimation derived in Part 1 (Construction 2), the area of the lemniscate associated with $P_k$ satisfies:
\[ \text{Area}(\{z : |P_k(z)| < 1\}) \le \frac{2\pi}{R_k^2 - 4}. \]
Since $\lim_{k \to \infty} R_k = \infty$, the area converges to 0.
Because $\mu(F)$ is the infimum of such areas, $\mu(F) = 0$.

\paragraph{Case 2: $F$ is Bounded}

If $F$ is closed and bounded, it is compact. Let $C = d_\infty(F)$. By hypothesis, $C \ge 1$.
Define the scaled set $K = \frac{1}{C}F = \{z/C : z \in F\}$.
Since capacity scales linearly, $d_\infty(K) = C^{-1} d_\infty(F) = 1$.
Thus, $K$ is a compact set with unit logarithmic capacity.
We now utilize the recent theorem concerning minimal lemniscate areas for such sets.

\begin{theorem*}[Krishnapur, Lundberg, Ramachandran, 2025]
Let $K \subset \mathbb{C}$ be a compact set with logarithmic capacity equal to 1. Let $\mathcal{A}_n(K)$ be the minimal area of $\{ z : |P(z)| < 1 \}$ over all monic polynomials $P$ of degree $n$ with roots in $K$. Then:
\[ \lim_{n \to \infty} \mathcal{A}_n(K) = 0. \]
More precisely, $\mathcal{A}_n(K) = O(1/\log \log n)$.
\end{theorem*}

Using this theorem, there exists a sequence of monic polynomials $\{q_n\}$ of degree $n$ with roots in $K$ such that $\text{Area}(\{|q_n| < 1\}) \to 0$.
Define polynomials on $F$ by $P_n(z) = C^n q_n(z/C)$.
Since $q_n$ has roots in $K$, $P_n$ has roots in $C \cdot K = F$. Furthermore, $P_n$ is monic.
Consider the lemniscate for $P_n$:
\[ \{ z : |P_n(z)| < 1 \} = \{ z : C^n |q_n(z/C)| < 1 \}. \]
Substitute $w = z/C$ (so $dA_z = C^2 dA_w$):
\[ \text{Area}(\{z : |P_n(z)| < 1\}) = C^2 \cdot \text{Area}(\{ w : |q_n(w)| < C^{-n} \}). \]
Since $C \ge 1$, we have $C^{-n} \le 1$. Thus:
\[ \{ w : |q_n(w)| < C^{-n} \} \subseteq \{ w : |q_n(w)| < 1 \}. \]
Consequently:
\[ \text{Area}(\{|P_n| < 1\}) \le C^2 \cdot \text{Area}(\{w : |q_n(w)| < 1\}). \]
As $n \to \infty$, the term $\text{Area}(\{|q_n| < 1\})$ tends to zero. Since $C$ is a fixed constant, the total area tends to zero.
Thus, $\mu(F) = 0$.

\section{Conclusion}

\begin{enumerate}
    \item \textbf{No.} $\mu(F)$ is not determined by the transfinite diameter. Even among sets with $d_\infty(F)=0$, the value of $\mu(F)$ can vary widely (e.g., $\ge \pi/4$ vs. $0$).
    \item \textbf{Yes.} If $d_\infty(F) \ge 1$, then $\mu(F) = 0$. Unbounded sets allow for quadratic polynomials with arbitrarily large root separations yielding zero area in the limit. Bounded sets rescaled to unit capacity utilize the theorem of Krishnapur, Lundberg, and Ramachandran to guarantee vanishing area.
\end{enumerate}

\begin{thebibliography}{9}
\bibitem{KLR25} M. Krishnapur, E. Lundberg, and K. Ramachandran, \textit{On the area of polynomial lemniscates}, arXiv:2503.18270, 2025.
\bibitem{EHP58} P. Erdős, F. Herzog, and G. Piranian, \textit{Metric properties of polynomials}, J. Analyse Math. 6 (1958), 125–148.
\end{thebibliography}
\end{solution}

\begin{problem}{User Prompt (Erd\H{o}s-1051)}
Is it true that if $a_1<a_2<\cdots$ is a sequence of integers with\[\liminf a_n^{1/2^n}>1\]then\[\sum_{n=1}^\infty \frac{1}{a_na_{n+1}}\]is irrational?
\end{problem}

\begin{solution}{Model Response (\emph{Aletheia})}
    \textbf{Introduction.} In this paper, we examine the arithmetic nature of the series
\[ S = \sum_{n=1}^\infty \frac{1}{a_n a_{n+1}}, \]
where $(a_n)_{n \ge 1}$ is a strictly increasing sequence of integers. We assume the sequence exhibits double exponential growth, quantified by the condition:
\begin{equation} \label{eq:hypothesis}
L := \liminf_{n\to\infty} a_n^{1/2^n} > 1.
\end{equation}
This problem relates to classical results by Erd\H{o}s, Graham, and Badea concerning the irrationality of series with rapidly growing denominators \cite{Erdos1980, Badea1987}. We prove that condition \eqref{eq:hypothesis} is sufficient to guarantee the irrationality of $S$.

\section{Main Result}

\begin{theorem}
Let $(a_n)_{n \ge 1}$ be a strictly increasing sequence of integers satisfying \eqref{eq:hypothesis}. Then the sum $S$ is an irrational number.
\end{theorem}

\begin{proof}
We proceed by contradiction. Assume that $S$ is a rational number.

\paragraph{Reduction to a Tail Sequence.}
Since $L > 1$, the terms $a_n$ tend to infinity. Let $c$ be a constant such that $1 < c < L$. By the definition of the limit inferior, there exists an index $N_0$ such that for all $n \ge N_0$, $a_n^{1/2^n} > c$. Consequently, $a_n \ge 2$ for all $n \ge N_0$. The original series $S$ can be decomposed into a finite sum and an infinite tail:
\[ S = \sum_{n=1}^{N_0-1} \frac{1}{a_n a_{n+1}} + \sum_{n=N_0}^\infty \frac{1}{a_n a_{n+1}}. \]
Since the finite sum is rational, $S$ is rational if and only if the tail sum is rational. We define a shifted sequence $(b_n)_{n \ge 1}$ by $b_n = a_{N_0 + n - 1}$. The sequence $(b_n)$ satisfies:
\begin{enumerate}
\item $b_n \in \mathbb{Z}$, $b_n \ge 2$, and $b_{n+1} > b_n$ for all $n \ge 1$.
\item The sum $S' = \sum_{n=1}^\infty \frac{1}{b_n b_{n+1}}$ is rational. Let $S' = p/q$ for positive integers $p, q$.
\item The growth rate is preserved in the limit. Let $\mathcal{L} = \liminf_{n \to \infty} b_n^{1/2^n}$. We have:
\[ \mathcal{L} = \liminf_{n \to \infty} \left( a_{N_0+n-1}^{1/2^{N_0+n-1}} \right)^{2^{N_0-1}} = L^{2^{N_0-1}}. \]
Since $L > 1$ and $N_0 \ge 1$, we have $\mathcal{L} > 1$.
\end{enumerate}

\paragraph{Integer Constraints from Rationality.}
Let $P_n = \prod_{k=1}^n b_k$. Define the partial sum $S'_n = \sum_{k=1}^{n-1} \frac{1}{b_k b_{k+1}}$ and the remainder $R_n = S' - S'_n$. Substituting $S' = p/q$:
\[ R_n = \frac{p}{q} - S'_n \implies q P_n R_n = p P_n - q (P_n S'_n). \]
The term $P_n S'_n = \sum_{k=1}^{n-1} \frac{P_n}{b_k b_{k+1}}$ is an integer because for every $k < n$, the distinct factors $b_k$ and $b_{k+1}$ divide $P_n$. Thus, the quantity $K_n := q P_n R_n$ is an integer. Since the terms of the series are strictly positive, $R_n > 0$, implying $K_n \ge 1$. This yields a lower bound for the remainder:
\begin{equation} \label{eq:Rn_lower}
R_n \ge \frac{1}{q P_n}.
\end{equation}

\paragraph{Recurrence Bound on Sequence Growth.}
We establish an upper bound for $b_{n+1}$ derived from the properties of $R_n$. Using the telescoping inequality $\frac{1}{b_k b_{k+1}} < \frac{1}{b_k} - \frac{1}{b_{k+1}}$, we sum from $k=n$ to infinity:
\[ R_n = \sum_{k=n}^\infty \frac{1}{b_k b_{k+1}} < \frac{1}{b_n}. \]
From $K_n = q P_n R_n$, we have:
\[ K_n < \frac{q P_n}{b_n} = q P_{n-1} \quad (\text{with } P_0=1). \]
Applying this to index $n+1$, we obtain $K_{n+1} < q P_n$. Next, we use the recurrence relation for the remainder:
\[ R_n = \frac{1}{b_n b_{n+1}} + R_{n+1}. \]
Multiplying by $q P_{n+1} = q P_n b_{n+1}$:
\[ q P_{n+1} R_n = \frac{q P_n b_{n+1}}{b_n b_{n+1}} + q P_{n+1} R_{n+1}. \]
Simplifying the first term on the right-hand side using $P_n/b_n = P_{n-1}$:
\[ b_{n+1} (q P_n R_n) = q P_{n-1} + (q P_{n+1} R_{n+1}). \]
Substituting $K_n$ and $K_{n+1}$:
\[ b_{n+1} K_n = q P_{n-1} + K_{n+1}. \]
Since $K_n \ge 1$, we have $b_{n+1} \le b_{n+1} K_n$. Therefore:
\[ b_{n+1} \le q P_{n-1} + K_{n+1}. \]
Using the bound $K_{n+1} < q P_n$:
\[ b_{n+1} < q P_{n-1} + q P_n = q P_{n-1} (1 + b_n). \]
Since $b_n \ge 2$, $1+b_n \le 2 b_n$. Thus:
\[ b_{n+1} < 2 q P_{n-1} b_n = 2 q P_n. \]
Substituting this into $P_{n+1} = P_n b_{n+1}$, we obtain the growth constraint:
\begin{equation} \label{eq:Pn_recurrence}
P_{n+1} < 2 q P_n^2.
\end{equation}

\paragraph{Convergence of Sequence Limits.}
We analyze the asymptotic behavior of $P_n$. Taking the natural logarithm of \eqref{eq:Pn_recurrence}:
\[ \ln P_{n+1} < 2 \ln P_n + \ln(2q). \]
Dividing by $2^{n+1}$:
\[ \frac{\ln P_{n+1}}{2^{n+1}} < \frac{\ln P_n}{2^n} + \frac{\ln(2q)}{2^{n+1}}. \]
Let $y_n = 2^{-n} \ln P_n$. Then $y_{n+1} < y_n + 2^{-(n+1)} \ln(2q)$. Consider the auxiliary sequence $z_n = y_n + 2^{-n} \ln(2q)$. We observe that $z_n$ is strictly decreasing. To prove convergence, we show it is bounded below. Since $\mathcal{L} > 1$, for sufficiently large $n$, $b_n > c^{2^n}$ for some $c > 1$. This implies $P_n$ grows at least double exponentially, so $y_n$ is bounded away from 0. Thus $z_n$ converges, implying $y_n$ converges. Let $Y = \lim_{n\to\infty} y_n$ and define:
\[ \Pi = \lim_{n\to\infty} P_n^{1/2^n} = e^Y. \]
We determine the limit of the normalized sequence terms. Since $b_n = P_n / P_{n-1}$:
\[ b_n^{1/2^n} = \frac{P_n^{1/2^n}}{ (P_{n-1}^{1/2^{n-1}})^{1/2} }. \]
Taking limits as $n \to \infty$:
\[ \lim_{n\to\infty} b_n^{1/2^n} = \frac{\Pi}{\sqrt{\Pi}} = \sqrt{\Pi}. \]
Since the limit exists, it must equal the limit inferior $\mathcal{L}$. Therefore:
\begin{equation} \label{eq:Pi_L2}
\Pi = \mathcal{L}^2.
\end{equation}

\subsection*{Contradiction via Remainder Estimation}

We now derive a conflicting lower bound for $\Pi$. Fix any constant $C$ such that $1 < C < \mathcal{L}$. For sufficiently large $k$, $b_k > C^{2^k}$. The term $b_k b_{k+1}$ grows as $C^{2^k} C^{2^{k+1}} = C^{3 \cdot 2^k}$. The remainder $R_n$ is bounded by:
\[ R_n = \sum_{k=n}^\infty \frac{1}{b_k b_{k+1}} < \sum_{k=n}^\infty C^{-3 \cdot 2^k}. \]
For $C > 1$, this series is dominated by its first term (bounded by a geometric series with ratio $\le 1/2$ for large $n$):
\[ R_n < 2 C^{-3 \cdot 2^n}. \]
Recall the rationality lower bound \eqref{eq:Rn_lower}, $R_n \ge \frac{1}{q P_n}$. Combining these:
\[ \frac{1}{q P_n} < 2 C^{-3 \cdot 2^n} \implies P_n > \frac{1}{2q} C^{3 \cdot 2^n}. \]
Raising to the power $1/2^n$ and taking the limit $n \to \infty$:
\[ \Pi \ge \lim_{n\to\infty} (2q)^{-1/2^n} C^3 = C^3. \]
Since this holds for all $C < \mathcal{L}$, we conclude:
\begin{equation} \label{eq:Pi_L3}
\Pi \ge \mathcal{L}^3.
\end{equation}

\subsection*{Conclusion}

We have derived two relations: $\Pi = \mathcal{L}^2$ and $\Pi \ge \mathcal{L}^3$. Combining them yields $\mathcal{L}^2 \ge \mathcal{L}^3$. Since $\mathcal{L} > 1$, we divide by $\mathcal{L}^2$ to obtain:
\[ 1 \ge \mathcal{L}. \]
This contradicts the condition $\mathcal{L} > 1$. Therefore, the assumption that $S$ is rational implies a contradiction.
\end{proof}

\begin{thebibliography}{9}
\bibitem{Erdos1980} P. Erd\H{o}s and R. L. Graham, \textit{Old and New Problems and Results in Combinatorial Number Theory}, Monographies de L'Enseignement Math\'ematique, No. 28, Universit\'e de Gen\`eve, Geneva, 1980.
\bibitem{Badea1987} C. Badea, ``The irrationality of certain infinite series,'' \textit{Glasgow Mathematical Journal}, vol. 29, no. 2, pp. 221--228, 1987.
\end{thebibliography}
\end{solution}

\begin{problem}{User Prompt (Erd\H{o}s-1089)}
Let $g_d(n)$ be minimal such that every collection of $g_d(n)$ points in $\mathbb{R}^d$ determines at least $n$ many distinct distances. Estimate $g_d(n)$. In particular, does\[\lim_{d\to \infty}\frac{g_d(n)}{d^{n-1}}\]exist?
\end{problem}

\begin{solution}{Model Response (\emph{Aletheia})}
\begin{theorem}
Let $n \ge 1$ be an integer. Let $g_d(n)$ denote the minimal integer such that every set of $g_d(n)$ distinct points in $\mathbb{R}^d$ determines at least $n$ distinct non-zero distances.
\begin{itemize}
\item For $n=1$, $g_d(1) = 2$, and the limit is $2$.
\item For $n \ge 2$, the limit exists and is equal to $\frac{1}{(n-1)!}$.
\end{itemize}
\end{theorem}

\begin{proof}
\textbf{1. Reduction to $s$-distance sets}

Let $P \subset \mathbb{R}^d$ be a finite set of points. The set of distinct non-zero distances determined by $P$ is defined as $D(P) = \{ \|x-y\| : x,y \in P, x \neq y \}$.
The definition of $g_d(n)$ implies that if $|P| \ge g_d(n)$, then $|D(P)| \ge n$.
Taking the contrapositive, if $|D(P)| \le n-1$, then $|P| < g_d(n)$.
Thus, $g_d(n) - 1$ is the maximum possible cardinality of a set $P \subset \mathbb{R}^d$ that determines at most $n-1$ distinct distances.
Let $s = n-1$. Let $M_d(s)$ denote the maximum size of an $s$-distance set in $\mathbb{R}^d$ (a set determining at most $s$ distinct distances).
Then:
\[ g_d(n) = M_d(s) + 1. \]

\textbf{2. Case $n=1$}

If $n=1$, then $s=0$.
A set with 0 distinct non-zero distances cannot contain any pair of distinct points. Thus, it contains at most 1 point.
So $M_d(0) = 1$.
Therefore, $g_d(1) = 1 + 1 = 2$.
The limit is:
\[ \lim_{d\to \infty} \frac{g_d(1)}{d^{1-1}} = \frac{2}{1} = 2. \]

\textbf{3. Case $n \ge 2$}

Here $s = n-1 \ge 1$. We estimate $M_d(s)$.

\textit{Upper Bound.}
We use the bound for $s$-distance sets in Euclidean space established by Bannai, Bannai, and Stanton (1983).
Let $X \subset \mathbb{R}^d$ be a set with $|D(X)| \le s$.
Let the distinct distances be $\alpha_1, \dots, \alpha_s$.
For each $y \in X$, define the polynomial $F_y: \mathbb{R}^d \to \mathbb{R}$ by:
\[ F_y(x) = \prod_{j=1}^s (\|x-y\|^2 - \alpha_j^2). \]
Notice that for any $z \in X$, $F_y(z) = 0$ if $z \ne y$, and $F_y(y) = \prod (-\alpha_j^2) \ne 0$.
This property implies that the polynomials $\{F_y\}_{y \in X}$ are linearly independent.
Expanding the expression $\|x-y\|^2 = \|x\|^2 - 2\langle x, y \rangle + \|y\|^2$, we observe that each $F_y(x)$ is a linear combination of terms of the form $(\|x\|^2)^k x_1^{e_1} \dots x_d^{e_d}$, where the total degree is at most $2s$.
Bannai et al. showed that these polynomials reside in a specific subspace of the polynomial ring associated with harmonic polynomials, which has dimension exactly $\binom{d+s}{s}$.
Since the polynomials are linearly independent, the number of points cannot exceed the dimension of this space.
Thus:
\[ |X| \le \binom{d+s}{s}. \]
Since this holds for any set with at most $s$ distances,
\[ M_d(s) \le \binom{d+s}{s}. \]
Substituting $s=n-1$:
\[ g_d(n) \le \binom{d+n-1}{n-1} + 1. \]

\textit{Lower Bound.}
We construct a set with a large number of points determining at most $s$ distances.
Consider the vector space $\mathbb{R}^{d+1}$. Let $V$ be the set of binary vectors with exactly $s$ ones (Hamming weight $s$):
\[ V = \left\{ v \in \{0,1\}^{d+1} : \sum_{i=1}^{d+1} v_i = s \right\}. \]
All points in $V$ satisfy the equation $\sum x_i = s$, defining a hyperplane $H \subset \mathbb{R}^{d+1}$. Since $H$ is a $d$-dimensional affine subspace, it is isometric to $\mathbb{R}^d$. Thus, $V$ is isometrically embeddable in $\mathbb{R}^d$.
The cardinality of $V$ is:
\[ |V| = \binom{d+1}{s}. \]
Now, let $u, v \in V$ be distinct points. The squared Euclidean distance is:
\[ \|u-v\|^2 = \sum_{i=1}^{d+1} (u_i - v_i)^2. \]
Since $u_i, v_i \in \{0,1\}$, $(u_i - v_i)^2 = |u_i - v_i|$.
Also, $\sum |u_i - v_i| = \text{weight}(u) + \text{weight}(v) - 2|u \cap v| = s + s - 2k = 2(s-k)$, where $k$ is the number of common ones.
Since $u \ne v$, the intersection size $k$ is an integer satisfying $0 \le k \le s-1$.
Thus, the possible squared distances are $\{ 2(s-k) : k = 0, 1, \dots, s-1 \}$.
There are at most $s$ distinct values (specifically $\sqrt{2}, \sqrt{4}, \dots, \sqrt{2s}$).
Therefore, $V$ is an $s$-distance set.
This implies:
\[ M_d(s) \ge |V| = \binom{d+1}{s}. \]
Substituting $s=n-1$:
\[ g_d(n) \ge \binom{d+1}{n-1} + 1. \]

\textbf{4. Calculating the Limit}

We combine the inequalities for $n \ge 2$:
\[ \binom{d+1}{n-1} + 1 \le g_d(n) \le \binom{d+n-1}{n-1} + 1. \]
Dividing by $d^{n-1}$:
\[ \frac{\binom{d+1}{n-1}}{d^{n-1}} + \frac{1}{d^{n-1}} \le \frac{g_d(n)}{d^{n-1}} \le \frac{\binom{d+n-1}{n-1}}{d^{n-1}} + \frac{1}{d^{n-1}}. \]
We evaluate the limits of the bounding terms as $d \to \infty$.
Using the expansion $\binom{N}{k} = \frac{N^k}{k!} + O(N^{k-1})$, we have:
\[ \lim_{d\to \infty} \frac{\binom{d+1}{n-1}}{d^{n-1}} = \lim_{d\to \infty} \frac{(d+1)\cdots(d-n+3)}{d^{n-1}(n-1)!} = \frac{1}{(n-1)!}. \]
Similarly,
\[ \lim_{d\to \infty} \frac{\binom{d+n-1}{n-1}}{d^{n-1}} = \lim_{d\to \infty} \frac{(d+n-1)\cdots(d+1)}{d^{n-1}(n-1)!} = \frac{1}{(n-1)!}. \]
Since the terms $1/d^{n-1}$ vanish as $d \to \infty$, both the upper and lower bounds converge to $\frac{1}{(n-1)!}$.
By the Squeeze Theorem,
\[ \lim_{d\to \infty}\frac{g_d(n)}{d^{n-1}} = \frac{1}{(n-1)!}. \]

\end{proof}

\textbf{Answer:}
For $n \ge 2$, the limit exists and equals $\frac{1}{(n-1)!}$.
For $n=1$, the limit is $2$.

\textbf{References}
\begin{enumerate}
\item Bannai, E., Bannai, E., \& Stanton, D. (1983). An upper bound for the cardinality of an s-distance subset in real Euclidean space. \textit{Combinatorica}, 3(2), 147–152.
\end{enumerate}
\end{solution}

\begin{problem}{User Prompt (Erd\H{o}s-1105)}
    The anti-Ramsey number $\mathrm{AR}(n,G)$ is the maximum possible number of colours in which the edges of $K_n$ can be coloured without creating a rainbow copy of $G$ (i.e. one in which all edges have different colours). Let $C_k$ be the cycle on $k$ vertices. Is it true that\[\mathrm{AR}(n,C_k)=\left(\frac{k-2}{2}+\frac{1}{k-1}\right)n+O(1)?\]Let $P_k$ be the path on $k$ vertices and $\ell=\lfloor\frac{k-1}{2}\rfloor$. If $n\geq k\geq 5$ then is $\mathrm{AR}(n,P_k)$ equal to\[\max\left(\binom{k-2}{2}+1, \binom{\ell-1}{2}+(\ell-1)(n-\ell+1)+\epsilon\right)\]where $\epsilon=1$ if $k$ is odd and $\epsilon=2$ otherwise?
\end{problem}

\begin{solution}{Model Response (\emph{Aletheia})}
    \begin{theorem}
For any integer $k \ge 3$, the anti-Ramsey number of the cycle $C_k$ satisfies
\[ \AR(n, C_k) = \left(\frac{k-2}{2} + \frac{1}{k-1}\right)n + O(1). \]
\end{theorem}

\begin{proof}
We affirm that the stated asymptotic formula is correct. This follows from the exact determination of $\AR(n, C_k)$ by Montellano-Ballesteros and Neumann-Lara [1], confirming a conjecture of Erdős, Simonovits, and Sós.

\paragraph{Exact Formula:}
Montellano-Ballesteros and Neumann-Lara proved that for all $n \ge k \ge 3$,
\[ \AR(n, C_k) = \left\lfloor \frac{n}{k-1} \right\rfloor \binom{k-1}{2} + \binom{n \bmod (k-1)}{2} + \left\lceil \frac{n}{k-1} \right\rceil. \]
Let $n = q(k-1) + r$, where $0 \le r < k-1$. Then $q = \frac{n-r}{k-1}$.
Substituting these terms into the exact formula:
\begin{align*}
\AR(n, C_k) &= q \binom{k-1}{2} + \binom{r}{2} + \left(q + \mathbb{I}_{r>0}\right) \\
&= q \left( \frac{(k-1)(k-2)}{2} + 1 \right) + \binom{r}{2} + \mathbb{I}_{r>0} \\
&= \frac{n-r}{k-1} \left( \frac{k^2 - 3k + 4}{2} \right) + O(1).
\end{align*}
Extracting the coefficient of $n$:
\[ \text{Coefficient} = \frac{k^2 - 3k + 4}{2(k-1)}. \]
Observe that:
\[ \frac{k-2}{2} + \frac{1}{k-1} = \frac{(k-1)(k-2) + 2}{2(k-1)} = \frac{k^2 - 3k + 4}{2(k-1)}. \]
Thus,
\[ \AR(n, C_k) = \left( \frac{k-2}{2} + \frac{1}{k-1} \right)n + O(1). \]



To provide context, we briefly describe the lower bound construction. Partition the vertex set $V(K_n)$ into sets $V_1, \dots, V_q$ of size $k-1$ and one remainder set. Assign distinct colors to all internal edges of each set. Order the sets and assign a single unique color $x_i$ to all edges connecting $V_i$ to any set succeeding it in the ordering. Any cycle $C_k$ must intersect at least two sets. If $V_i$ is the first set visited by the cycle, the cycle must traverse at least two edges connecting $V_i$ to later sets; these edges share the color $x_i$, preventing the cycle from being rainbow.
\end{proof}


\begin{theorem}
Let $P_k$ be a path on $k$ vertices and $\ell = \lfloor \frac{k-1}{2} \rfloor$. For all integers $n \ge k \ge 5$, the anti-Ramsey number $\AR(n, P_k)$ is given exactly by:
\[ \AR(n, P_k) = \max\left(\binom{k-2}{2}+1, \binom{\ell-1}{2}+(\ell-1)(n-\ell+1)+\epsilon\right), \]
where $\epsilon=1$ if $k$ is odd and $\epsilon=2$ if $k$ is even.
\end{theorem}

\begin{proof}
We affirm that the statement is correct. The exact value of $\AR(n, P_k)$ for $n \ge k \ge 5$ was established by Yuan [2]. We rigorously verify the lower bounds corresponding to the two expressions in the maximization and cite the upper bound result.

Let $f_1(n, k) = \binom{k-2}{2} + 1$ and $f_2(n, k) = \binom{\ell-1}{2} + (\ell-1)(n-\ell+1) + \epsilon$.

\paragraph{Lower Bound Construction 1 (Value $f_1$):}
Select a subset $S \subset V(K_n)$ of size $k-2$.
Define the edge coloring $c$ as follows:
\begin{enumerate}
    \item Assign a distinct unique color to every edge in the induced subgraph $G[S]$. (Total: $\binom{k-2}{2}$ colors).
    \item Assign a single color $c_0$ to all other edges (edges in $V \setminus S$ and edges between $S$ and $V \setminus S$).
\end{enumerate}
Total colors used: $\binom{k-2}{2} + 1$.

\textit{Verification:} Suppose there exists a rainbow path $P$ with $k$ vertices (length $k-1$). Since all edges incident to $V \setminus S$ share the color $c_0$, a rainbow path can contain at most one such edge. Consequently, at least $(k-1) - 1 = k-2$ edges of $P$ must lie entirely within $S$. However, the subgraph induced by $S$ has only $k-2$ vertices. The longest path in a graph with $k-2$ vertices has length $k-3$. Thus, it is impossible to find $k-2$ edges within $S$. This contradiction proves no rainbow $P_k$ exists.

\paragraph{Lower Bound Construction 2 (Value $f_2$):}
Partition $V(K_n)$ into two sets $A$ and $B$ such that $|A| = \ell - 1$ and $|B| = n - \ell + 1$.
Define the edge coloring $c$ as follows:
\begin{enumerate}
    \item Assign distinct unique colors to every edge incident to at least one vertex in $A$ (i.e., edges in $A \times A$ and $A \times B$).
    The number of such edges is $\binom{|A|}{2} + |A||B| = \binom{\ell-1}{2} + (\ell-1)(n-\ell+1)$.
    \item Color the edges within the induced subgraph $G[B]$ using exactly $\epsilon$ colors:
    \begin{itemize}
        \item If $k$ is odd ($\epsilon=1$), assign all edges in $G[B]$ the same color $c_1$.
        \item If $k$ is even ($\epsilon=2$), partition the edges of $G[B]$ into two nonempty sets and assign colors $c_1, c_2$ respectively.
    \end{itemize}
\end{enumerate}
Total colors used: $\binom{\ell-1}{2} + (\ell-1)(n-\ell+1) + \epsilon$.



\textit{Verification:} Suppose there exists a rainbow path $P$ with $k$ vertices (length $k-1$).
Let $E_P$ denote the edge set of the path. We partition $E_P$ into edges incident to $A$ ($E_A$) and edges entirely within $B$ ($E_B$).
Every vertex in $A$ has degree at most 2 in the path $P$. Thus, the total number of edges in $P$ incident to vertices in $A$ satisfies $|E_A| \le \sum_{v \in A} \deg_P(v) \le 2|A| = 2(\ell - 1)$.
Since $|E_P| = k-1$, the number of edges required from $B$ is at least:
\[ |E_B| \ge (k-1) - 2(\ell - 1). \]

\textbf{Case 1: $k$ is odd.}
Here $\ell = \frac{k-1}{2}$.
Required $|E_B| \ge (k-1) - 2(\frac{k-1}{2} - 1) = k - 1 - (k - 3) = 2$.
However, all edges in $G[B]$ share a single color ($\epsilon=1$). Thus, any set of 2 edges in $B$ is monochromatic, not rainbow. A rainbow path can contain at most 1 edge from $B$. Contradiction.

\textbf{Case 2: $k$ is even.}
Here $\ell = \frac{k-2}{2}$.
Required $|E_B| \ge (k-1) - 2(\frac{k-2}{2} - 1) = (k-1) - (k-4) = 3$.
We have used only $\epsilon=2$ colors in $B$. By the Pigeonhole Principle, any selection of 3 edges from $B$ must contain at least two edges of the same color. Thus, a rainbow path can contain at most 2 edges from $B$. Contradiction.

In both cases, the constructed colorings avoid a rainbow $P_k$. Hence, $\AR(n, P_k) \ge f_2(n, k)$.

\paragraph{Upper Bound and Conclusion:}
Combining the two constructions yields $\AR(n, P_k) \ge \max(f_1, f_2)$.
Yuan [2] proved using stability analysis of the Erdős-Gallai Theorem that for $n \ge k \ge 5$, any edge coloring of $K_n$ using strictly more than this number of colors must contain a rainbow path $P_k$. Specifically, Yuan proved that extremal colorings must be structurally isomorphic to one of the two constructions detailed above.
Thus, the equality holds.
\end{proof}

\begin{thebibliography}{9}
\bibitem{1} J. J. Montellano-Ballesteros and V. Neumann-Lara, ``An anti-Ramsey theorem on cycles,'' \textit{Graphs and Combinatorics}, vol. 21, no. 3, pp. 343--354, 2005.
\bibitem{2} L. Yuan, ``The anti-Ramsey number for paths,'' \textit{Journal of Graph Theory}, vol. 102, no. 3, pp. 353--367, 2023.
\end{thebibliography}
\end{solution}
\end{document}
